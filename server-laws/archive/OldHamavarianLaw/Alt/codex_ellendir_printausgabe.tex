\documentclass{article}
\usepackage[utf8]{inputenc}
\usepackage{textcomp}
\usepackage{amsmath}
\usepackage{enumerate}
\usepackage{fontspec}

\renewcommand{\thesection}{\Roman{section}. Abschnitt:}
\counterwithout{subsection}{section}
\renewcommand{\thesubsection}{\Roman{subsection}}
\setmainfont{[Leipzig Fraktur Normal.ttf]}


\title{Codex Ellendir}
\author{Imperiale Administration}
\date{Februar 1185}

\begin{document}
\maketitle
\vspace*{\fill}
\paragraph{Hamavarisches Recht gemäß Recht nach HSGM Kaiser Ellendir IV. Daimitrayon Ellendir}

\newpage
\section{Feudalordnung}
\subsection{Feudalstruktur}
a.  Dem Staate unterstehen jegliches Territorium auf dessen Gebiet und die Einwohner, sowie Besucher jenes Gebietes.\\
b. Die Pflicht jedes Menschen auf dem Territorium des Staats ist es, den Befehlen des Kaisers ausnahmslos folgezuleisten.\\
c. Eine Ausnahme zu Absatz 2 bildet das Lex Accusantis\footnote{Das Lex Accusantis (Recht des Zweifelnden) entsprang einem Vorfall, bei welchem der Bruder Ellendirs III. vor Gericht stand, weil er den Befehl des Kaisers verweigert hatte. Aus Gründen der Nachvollziehbarkeit der Argumentation gewährte Ellendir III. das Lex Accusantis, welches in der Verordnung A001 steht. Der Begriff wurde zu einem Fachterminus in der hamavarischen Rechtslehre.}

\subsection{Der Kaiser}
a. Der Kaiser ist das Staatsoberhaupt und der Regierungschef des Staates Hamavar und daher in der Hierarchie an oberster Stelle. \\ 
b. Jedes Mitglied der Nation ist ihm zu Treue verpflichtet.\\
c. Er regiert das Kaiserreich und verfügt daher über absolute Entscheidungsvollmachten.\\
d. Der Kaiser steht über dem Gesetz.\\
e. Der Kaiser verfügt sowohl über das Besitz- als auch Verwaltungsrecht seiner Domänen.\\
f. Die direkte Anrede lautet "Eure erhabene und glorreiche Majestät".\\
g. Die indirekte Anrede lautet "Seine erhabene und glorreiche Majestät" und kann ebenfalls mit HSGM abgekürzt werden.\\
h. Die indirekte Adressierung verstorbener Kaiser lautet "His Late Glorious and Sublime Majesty", auch als "HLSGM" abgekürzt.\\
i. Im rechtlichen Kontext ist der Begriff des Kaisers Synonym mit dem Begriff Hamavars und dessen weitere Synonyme.

\subsection{Der Lordkanzler}
Der Lordkanzler fungiert als Oberhaupt der hamavarischen Verwaltung und sitzt daher der Imperialen Administration vor. Er ist in der Lage, den Staatsschatz zu verwalten. \\
a. Die direkte Anrede des Lordkanzlers lautet "Eure erkorene/erwählte Hoheit".\\
b. Die indirekte Anrede lautet "Seine/Ihre erkorene/erwählte Hoheit", auch abgekürzt als "HEH".\\
c. Der Lordkanzler steht über den Königen.

\subsection{Der Königsrat }
Als Königsrat wird die Versammlung bezeichnet, die aus dem Kaiser und den Ratskönigen (\ref{koenige}e) besteht. Sie fungiert nicht als Parlament, da die Ratsabstimmungen lediglich beratender Funktion sind und daher nicht vom Kaiser berücksichtigt werden müssen.  

\subsection{Der Landadel}
Als Landadel werden die Angehörigen jener Adelstitel bezeichnet, die über das Besitzrecht ihrer Domäne verfügen.\\
a. Hierzu gehören: \\
a. Die Hochkönige und Könige  \\
b. Die Fürsten  \\
c. Die Superioren und Grafen  \\
b. Bis auf Superioren dürfen alle Angehörigen des Landadels in die Entscheidungen des Verwaltungsadels auf ihrer Ebene eingreifen. Lediglich bei Legaten und Superioren ist dies umgekehrt.

\subsection{Der Verwaltungsadel}
Dem Verwaltungsadel gehören die Inhaber von Titeln an, die über das Verwaltungsrecht ihrer Domäne verfügen.  \\
a. Hierzu gehören:
\begin{enumerate}[a)]
	\item Die Imperialen Administratoren
	\item Die Magistraten
	\item Die Legaten und Hochlegaten
\end{enumerate}

\subsection{Der Hochadel}
Dem Hochadel gehören der gesamte Land- und Verwaltungsadel der Kaiserreichs-, Königreichs- und Fürstentumsebene an.

\subsection{Die Könige}\label{koenige} 
a. Ein König besitzt auf seinem Territorium uneingeschränkte Entscheidungsvollmachten. Er ist dazu verpflichtet, dem Kaiser Tribut zu zollen.  \\
b. Der Titel besteht lebenslänglich und ist erblich.  \\
c. Die Könige sind in der Lage, ihre Vasallen selbst zu wählen.  \\
d. Zum Hochkönig werden lediglich die Oberhäupter der Häuser Ellendir und Hèturrir erhoben. Ihnen allein steht das Privileg zu, das Hochkönigreich Hamavar zu regieren.\\
e. Angehörige des Königsrats werden als Ratskönige bezeichnet. Zu ihnen gehören die folgenden Könige:  \\
\begin{enumerate}[a)]
\item Die Hochkönige von Hamavar  
\item Der König von Mezavar  
\item Der König von Duumarkng  
\item Der König von Lúinna  
\item Der König von Lorrva  
\item \textit{Aufgrund eines Beschlusses des Hohen Tribunals unter Kaiser HLSGM Ellendir III. am 19. April 1161 wurden die Könige von Morrvarnirid auf unbegrenzte Zeit aus dem Rat suspendiert.} Dies fällt weg, da das Königreich Morrvarnirid durch einen Beschluss HSGM Kaiser Ellendirs IV. am 18. Januar 1180 aufgelöst wurde.
\end{enumerate}
g. Der König von Duumarkng hält zusätzlich den Titel des Befehlshabers der Armee inne.  \\
h. Der König von Lúinna hält zusätzlich den Titel des Schatzmeisters inne.  \\
i. Die direkte Anrede des Königs lautet "Eure königliche Hoheit".  \\
j. Die indirekte Anrede lautet "Seine/Ihre königliche Hoheit", auch mit "HRH" abgekürzt.

\subsection{Die Fürsten}
a. Fürsten verwalten ihr Territorium, müssen ihren König jedoch bei Entscheidungen um Erlaubnis bitten.  \\
b. Sie müssen ihrem Lehnsherrn Tribut zollen.  \\
c. Die direkte Anrede der Fürsten lautet "Euer Gnaden". \\
d. Die indirekte Anrede lautet "Seine/Ihre Gnaden", welche auch durch das Akronym "HG" ersetzt werden kann.  

\subsection{Die Superioren}
a. Der Titel des Superiors ist der niedrigste Titel des Landadels. Er befindet sich im Besitz eines Superioriats beziehungsweise Reichsbezirks. In Entscheidungsfragen müssen sie sich das Einverständnis ihres Fürste einholen.  \\
b. Als Grafen werden jene Superioren bezeichnet, die ein Gebiet besitzen, welches sich 1113 innerhalb der Fürstentümer Dimárva, Daimitra, Harrva und Thrrannúmenvar befand.  \\
c. Grafen werden als "Lord" adressiert.  

\subsection{Imperiale Administration}
a. Die Imperiale Administration verwaltet das Kaiserreich.  \\
b. Ihr sitzt der Lordkanzler vor.  \\
c. Eine Kaiserliche Kanzlei verwaltet ihren Administrationsbezirk.  \\
d. Ihnen sitzen die Imperiale Administratoren vor.  \\
e. Das Kaiserreich verfügt über zwei Kaiserliche Kanzleien, deren Standorte und Verwaltungsregionen die folgenden sind:  
\begin{enumerate}[a)]
\item Hrrátim: Das Hochkönigreich Hamavar, sowie die Königreiche Mezavar, Lorrva und Lúinna  
\item Illúthrrin: Die Königreiche Duumarkng, Morrvar und Varrendkhatar, sowie die Sonderverwaltungszone Fangrothva.
\end{enumerate}

\subsection{Magistratur}
a. Die Magistratur verwaltet ihr jeweiliges Königreich.  \\
b. Ihr sitzt das Magistratenkonzil vor.

\subsection{Die Legaten}
a. Die Legaten verwalten ihren jeweiligen Reichsbezirk und dessen Armee.  \\
b. Als Hochlegaten werden jene Legaten bezeichnet, die einer Reichsstadt vorsitzen. 

\subsection{Reichsbezirke}
Als Reichsbezirk wird der Zusammenschluss von Superioriaten ohne Legat und dem zentralen Superioriat mit Legat bezeichnet.  

\subsection{Reichsstädte}
a. Als Reichsstädte werden jene Städte bezeichnet, die von dem Kaiser diesen Status erhalten haben. Es wird an die drei zentralen Städte des Reichs vergeben.  \\
b. Derzeit verfügen zwei Städte über eine ständige Nominierung:  
\begin{enumerate}[a)]
\item Hrrátim als Reichshauptstadt  
\item Illúthrrin als Sitz des Kaisers  
\end{enumerate}

\subsection{Sonderrechte}
a. Dem Kaiser ist es gestattet, durch die Anwendung des Lex Votum Imperatoris jegliche Entscheidung jeglicher Instanz aufzuheben.  \\
b. Den Vasallen wird im Gegenzug gestattet, das Lex Accusantis anzuwenden. Laut diesem steht es ihnen frei, einen Befehl zu verweigern, sofern dieses Recht sofort angewendet wird. Beschließt der Lehnsherr, dass dieser Befehl dennoch ausgeführt werden soll, so muss man diesem Befehl dennoch folgeleisten. 

\subsection{Die Kaiserherrschaft}
a. Der Kaiser regiert uneingeschränkt bis zu seinem Lebensende und wählt vor seinem Tod einen Nachfolger aus seiner Linie.  \\
b. Die Autorität des Kaiser darf nicht angezweifelt werden.  \\
c. Der Kaiser kann Ausnahmen zu allen Gesetzen machen.  \\

\subsection{Erblicher Adel}
Erblicher Adel bezeichnet Adelstitel, die erblich sind und somit keine Beschränkung durch Legislatur erfahren. Dennoch können sie durch das Hohe Tribunal bei Veruntreuung oder sonstigem Missbrauch von Machten, die durch den Titel kommen, entzogen werden.  

\subsection{Amtlicher Adel}
Zum amtlichen Adel gehören Grafentitel und sonstige Ermächtigungen, die regelmäßig an die Inhaber bestimmter Ämter verliehen werden. Daher ist dieser Titel nicht erblich und zudem noch durch die Legislatur eingeschränkt.  

\subsection{Titularadel}
Titularadel bezeichnet Adelstitel, die keine Ermächtigungen haben, jedoch als Auszeichnung dienen. Hierzu gehören neue Familiennamen oder ein Namenszusatz für eine Person. Die Erblichkeit hängt von der Art des Titels ab.  

\subsection{Staatsbürger}
a. Als Staatsbürger werden die Einwohner von Hamavar bezeichnet, denen eine Staatsbürgerschaft gewährt wurde. \\  
b. Die Staatsbürger sind in zwei Klassen unterteilt:  \\
\begin{enumerate}[a)]
\item Als Itagoren bezeichnet man Bürger, deren Familie offiziell als rein-liurnosisch gilt und in Hamavar geboren wurden. Auch gelten Bürger als Itagoren, die halb-hamavarischen Blutes sind, dennoch aber in Hamavar geboren wurden.  
\item Als Gadigoren bezeichnet man Fremdbürger, sprich Bürger, die nicht den Status des Itagoren besitzen. Familien, die seit über acht Generationen den Status der Gadigoren innehalten, werden zu Itagoren erhoben. Hierbei darf jedoch keine Generation einen Großteil außerhalb Hamavars gelebt haben. Trifft dies jedoch zu, so muss erneut acht Generationen gewartet werden.  
\end{enumerate}

\section{Struktur der Judikative}

\subsection{Gerichtliche Instanzen}
a. Erhebt eine Partei Anklage, so beginnt der Rechtsstreit in der untersten Instanz. Sofern man gemäß \ref{verlauf}j in Berufung gegangen ist, wird das Verfahren von der nächsten Instanz behandelt.\\
b. Die Instanzen in aufsteigender Folge sind:
\begin{enumerate}[a)]
	\item Atilengericht
	\item Fürstenkammer
	\item Königskammer
	\item Hohes Tribunal
\end{enumerate}
c. Der Kaiser kann nach eigenem Ermessen den Richterschaftsvorsitz eines Prozesses jederzeit übernehmen.

\subsection{Atilengericht}
Das Atilengericht ist für Rechtsstreitigkeiten auf der Reichsbezirksebene verantwortlich. Ihm sitzt der jeweilige Legat, beziehungsweise Hochlegat vor.

\subsection{Fürstenkammer}
Die Kammer des Fürsten ist für Rechtsstreitigkeiten auf Fürstentumsebene zuständig. Ihr sitzt der jeweilige Fürst vor.

\subsection{Königskammer}
Der Königskammer sitzt der jeweilige König vor. Dementsprechend ist sie auf der Königreichsebene tätig.

\subsection{Hohes Tribunal}
Das Hohe Tribunal ist der höchste Gerichtshof Hamavars und besteht aus dem Königsrat. Sein Vorsitzender ist der Kaiser.

\subsection{Zeugen}\label{zeugen}
Man darf Personen in den Zeugenstand berufen.\\
a. Diese darf man unter den gegebenen Regeln befragen  \\
b. Diese Regeln lauten:  \\
\begin{enumerate}[a)]
	\item Die Zeugen stehen automatisch unter Eid, sobald sie ihr erstes Wort im Zeugenstand erheben.
	\item Die Zeugen müssen daher alles wahrheitsgemäß beantworten.
	\item Jegliche ungenauen Aussagen der Zeugen werden nicht ins Protokoll aufgenommen (siehe hierzu \ref{eordnung}e).
\end{enumerate}

\subsection{Anwälte}
Man darf einen Anwalt einstellen. Hierbei muss jedoch beachtet werden, dass kein Anrecht auf einen Pflichtverteidiger besteht.

\subsection{Einspruchsordnung}
\label{eordnung}
a. Einsprüche sind erlaubt und bilden eine Ausnahme zu \ref{gordnung}a.\\
b. Sie können durch die Richterschaft abgewiesen werden.\\
c. Bei einmaliger Ablehnung eines Einspruchs darf dieser nicht auf dieselbe Aussage erneut angewandt werden.\\
d. Auf die Ankündigung eines Einspruchs muss stets die Ankündigung des Grundes folgen.\\
e. Rechtlich zulässige Gründe sind:
\begin{enumerate}[a)]
\item Nicht aussagekräftig/unverständlich/mehrdeutig: Die Aussage oder Frage ist aufgrund seiner nicht aussagekräftigen Natur unzulässig.
\item Bereits beantwortet: Die gleiche Frage wurde mehrfach gestellt, obwohl sie bereits beantwortet wurde.
\item Unbewiesene Vermutung: Der Anwalt behauptet etwas, ohne sich auf vorliegende Beweise zu stützen.
\item Fordert Spekulationen: Der Anwalt fordert den Zeugen auf, zu spekulieren.
\item Supra interrogatio (über Befragung hinaus): Der Anwalt fragt mehr als eine Frage gleichzeitig.
\item Mangelnde Kenntnisse: Die Kenntnisse des Zeugens über das gefragte Thema sind unzureichend nachgewiesen.
\item Ohne Priorität: Die Frage ist dem Prozess beziehungsweise der Befragung nicht dienlich.
\item Gerücht: Die Antwort der Partei baut auf außergerichtlichen Aussagen auf.
\item Lex Accusantis: Ein Mitglied des Hohen Tribunals hat einen Einwand gegen eine Entscheidung des Kaisers.\\
i. Dieser Einspruch kann nur durch Mitglieder des Hohen Tribunals getätigt werden, die als richtende Partei im Prozess dienen.\\
\item Hinterfragt die Staatsautorität: Eine Partei fechtet, hinterfragt oder beleidigt die Staatsautorität beziehungsweise die Autorität des Kaisers. Wird dieser Einspruch bewilligt, wird derjenige, der die Aussage gebracht hat, hinterher wegen Verstoßes gegen LI vor Gericht gestellt.
\end{enumerate}
f. Wird ein Einspruch stattgegeben, so muss der Befragende bei der Befragung mit der nächsten Frage fortfahren. Der Zeuge darf die vorherige Frage nicht beantworten oder seine Aussage wird im Fall, dass er sie bereits getätigt hat oder dennoch antwortet, gestrichen. Erhebt ein Richter diesen Einspruch, so ist dem sofort stattgegeben, sofern der Gerichtsvorsitzende dem nicht widerspricht.

\subsection{Prozessverlauf}\label{verlauf}
Das hamavarische Recht sieht den nachfolgenden Verlauf für Gerichtsverfahren vor.\\
a. Alle Parteien mit Ausnahme der Richterschaft betreten den Raum.\\
b. Die Richterschaft versammelt sich. Währenddessen muss jeder Anwesende stehen.\\
c. Der Gerichtsvorsitzende eröffnet den Prozess und die weiteren Richter setzen sich.\\
d. Der Gerichtsvorsitzende verliest die Anklageschrift.\\
e. Der Kläger muss den Strafbestand aus seiner Sicht darlegen.\\
f. Der Beklagte hat das Wort und darf seine Darstellung des Sachverhalts darlegen.\\
g. Von nun an entscheidet der Gerichtsvorsitzende, wer das Wort erhält.\\
h. Sobald alle Beweise und Aussagen der beiden Parteien dargelegt wurden, dürfen die beklagte Partei und die klagende Partei, beziehungsweise deren Vertreter, je ein Strafmaß, beziehungsweise den Freispruch, empfehlen.\\
i. Die Richterschaft tritt zurück und berät sich in einem separaten Gespräch. Hierbei wird über die Strafe beratschlagt und anschließend entschieden. Bei Stimmgleichheit verfügt der Gerichtsvorsitzende eine zweite Stimme.\\
j. Die Richterschaft betritt den Saal, wobei erneut jeder stehen muss, und verkündet im Anschluss die Strafe. Daraufhin fragt der Gerichtsvorsitzende, ob eine Partei in Berufung gehen möchte, sofern denn eine höhere Instanz besteht. Andernfalls ist die Strafe final.\\
k. Bis der letzte Richter den Saal verlassen hat müssen alle Teilnehmer stehen und dürfen den Saal nicht verlassen.

\subsection{Gerichtsordnung}\label{gordnung}
a. Man darf nicht unaufgefordert sprechen\\
b. Man muss sich für den Prozess angemessen kleiden. Dementsprechend dürfen die Anwesenden keine Kopfbedeckungen mit sich führen und müssen einen Anzug in einer angemessenen Farbe tragen.
c. Richter müssen schwarze Anzüge tragen.\\
d. Im Falle des Hohen Tribunals müssen die Richter rote Anzüge tragen.\\ 
e. Verstöße gegen die Gerichtsordnung unter Inbezugnahme von \ref{zeugen} und \ref{verlauf} werden, sofern bereits eine Verwarnung erteilt wurde mit 10 HTK Bußgeld geahndet. Liegen nach Ermessen der Richterschaft zu viele Verstöße vor, können sie die schuldige Partei ungeachtet ihrer Relevanz für diesen Prozess aus dem Saal verweisen und das Verfahren anschließend in dessen Abwesenheit fortfahren.\\
f. Von b ist lediglich der Kaiser ausgenommen.

\subsection{Gerichtliche Vorladung}
Sofern ein Verfahren bestätigt wurde kann unter Vereinbarung mit beiden Parteien ein Gerichtstermin festgelegt werden. Dies wird als außerordentliche Vorladung angesehen.\\
a. Legt das Gericht einen Termin fest, so muss dieses beide Parteien in einem Schreiben deutlich über das Verhandlungsdatum informieren. Hierbei handelt es sich um eine ordentliche Vorladung\\
b. Der Termin und Ort einer Verhandlung muss spätestens zwölf Stunden vor Prozessbeginn bekanntgegeben werden.\\
c. Ein Antrag auf Aufschub kann bis zu zwei Stunden vor Prozessbeginn eingereicht werden.\\
d. Wird diesem Antrag durch den Gerichtsvorsitzenden des Verfahrens stattgegeben, so wird das Verfahren vertagt.\\
e. Andernfalls, oder wenn kein Antrag besteht, müssen die Parteien erscheinen, ansonsten wird in ihrer Abwesenheit verhandelt.\\
f. Erscheint keine Partei, so wird der Termin ebenfalls vertagt.\\
g. Jeder gemäß e abwesenden Partei droht eine Bußgeldstrafe in Höhe von 20 HTK.\\
h. Der Verhandlungsort wird gemäß Wohnsitz der beklagten Partei entschieden.\\
i. Verfügt die beklagte Partei über keinen Wohnsitz in Hamavar, so wird gemäß Wohnsitz der klagenden Partei entschieden.\\
j. Können weder i noch j erfüllt werden, so entscheidet der Staat über den Verhandlungsort.

\subsection{Anrede des Richters}
Steht man vor Gericht, so hat man den Richter als 'Euer Ehren' anzureden. Tut man dies nicht, muss man fünf HTK zahlen.\\
a. Steht man vor dem Hohen Tribunal, so hat man den Kaiser als 'Eure erhabene und glorreiche Majestät' anzureden und die weiteren Richter mit der Anrede, die ihnen zusteht, sofern sie höheren Ranges sind. Verweigert man dies, muss man 10 HTK zahlen und wird hingerichtet.\\
b. In beiden Fällen muss man erst nach einer Aufforderung dem nachkommen.

\subsection{Rechtliche Immunität}
a. Mangelnde oder fehlerhafte Kenntnisse des Gesetzes gewähren keine rechtliche Immunität, da das Informieren über die Gesetzeslage Pflicht ist.\\
b. Der Kaiser darf Personen rechtliche Immunität verleihen.

\subsection{Vergehen am Hochadel}\label{vergehen}
a. Vergehen an dem Hochadel werden mit dem dreifachen Strafsatz vergolten.\\
b. Vergehen an dem Kaiser werden mit dem zehnfachen Strafsatz vergolten.\\
c. Vergehen an dem Staat gelten als Vergehen an dem Kaiser.

\subsection{Verbannung}
a. Verbannung dient im Falle von Zahlungsunfähigkeit als Ersatz für hohe Bußgeldstrafen. Die verzehnfachte Bußgeldstrafe entspricht der Anzahl der Tage einer Verbannung.\\
b. Als Verbannter darf man das Gebiet des Kaiserreichs nicht betreten.

\subsection{Entschädigungssteuern}
a. Auf Entschädigungen werden zusätzlich zu den, im Recht fixierten Bußgeldsätzen, eine Steuer erhoben.\\ 
b. Der Steuersatz wird alle 30 Tage von dem Schatzmeister festgelegt.\\
c. Die Steuern umfassen einen Mindestbetrag von 1 HTK und werden stets aufgerundet.

\subsection{Freiheitsstrafe}
a. Eine Haftstrafe kann bei Beschluss des Gerichts entweder als Strafersatz oder Strafzusatz angewendet werden.\\
b. Bei Ausbruchsversuchen und Ausbrüchen werden stets zehn Minuten zusätzliche Haft angeordnet.\\
c. Beihilfe bei Ausbrüchen werden mit dem Verordnen der gleichen Haftstrafe für die helfende Partei bestraft.\\
d. Abgesessen hat man die Strafe, sobald man die jeweilige Zeit nachweislich online war.\\
e. Der Staat haftet für keine Gegenstände, die während der Haftstrafe verlorengehen, sofern für den Häftling genügend Zeit bestand, die Gegenstände anderweitig zu lagern.\\
f. Der Strafsatz bemisst sich in 5-Minuten-Sätzen

\subsection{Hinrichtung}
a. Hinrichtungen sind als Strafmaßnahme für Kapitalverbrechen vorgesehen.\\
b. Hinrichtungen sind erst dann erlaubt, wenn das Gericht eindeutig eine Hinrichtung verhängt hat.

\subsection{Verbindlichkeit von Strafsätzen}
a. Die aufgeführten Strafsätze dienen lediglich zur Orientierung und sind daher nicht verpflichtend.\\
b. Dies gilt nicht für Hinrichtungen.\\
c. Bei Wiederholungstaten liegt es je nach Häufigkeit und Schwere der Tat im Ermessen des zuständigen Gerichts, ob weiterhin derselbe oder ein verhärteter Strafsatz geltend gemacht werden sollte.\\
d. Bei äußerster Häufigkeit oder relativer Häufigkeit von Taten besonderer Schwere, haben Wiederholungstaten die Todesstrafe zur Folge.

\subsection{Untersuchungshaft}
Besteht die Gefahr, dass ein Tatverdächtiger bis zu seinem Prozess flieht oder befragt werden muss, muss eine Unterbringung in der Untersuchungshaft angeordnet werden.\\

\subsection{Unterbringung in Hochsicherheitseinrichtungen}
a.	Freiheitsstrafen in Höhe von mehr als zwanzig Minuten müssen in Hochsicherheitseinrichtungen abgesessen werden.\\
b.	Besteht eine akute Fluchtgefahr, so kann dies auch bei kürzerer Haft angeordnet werden.

\subsection{Unterbringung in einer Sonderverwahrung}
Personen, die sich eines Kapitalverbrechens schuldig gemacht haben und daher hingerichtet werden sollen, müssen sofern zusätzlich eine Freiheitsstrafe angeordnet wurde, in einer Todeszelle untergebracht werden. Mit Ende ihrer Haftstrafe werden sie hingerichtet\footnote{Vgl. CP-01/80: Hamavar gg. Marrkan-Bennetal}.

\subsection{Entzug von Titeln}
Es ist dem Hohen Tribunal gestattet, bestimmten Personen den Titel zu entziehen, sofern sie dessen Macht missbrauchen oder mit ihr anderweitig nicht umgehen können.

\subsection{Präzedenzfälle}
Sofern ein rechtlicher Ausnahmefall vorliegt, ist der Fall unter sofortiger Wirkung dem Hohen Tribunal zu übertragen.\\
a. Entscheidet dieser, dass es sich bei dem vorliegenden Fall um eine Straftat handelt, so muss dies umgehend in die Gesetze aufgenommen werden und
sofern nach Ermessen des Hohen Tribunals ein Bewusstsein des Verstoßes gegen moralische Normen durch die Beklagte vorliegen sollte, der Strafe entsprechend
geurteilt werden.

\subsection{Generationenrecht}
a. Verstirbt ein Kläger oder Opfer eines Verbrechens, so darf das Haus des Geschädigten Anklage erheben oder die Geschädigte vor Gericht vertreten.\\
b. Verstirbt ein Täter, so muss sich das Haus des Täters für dessen Straftaten verantworten.\\
c. Das Haus wird stets durch dessen Oberhaupt vertreten. Besteht keins, so wird dieses vom zuständigen Gericht gewählt.\\
d. Gemäß b können demnach auch die nachfolgenden Oberhäupter zur Rechenschaft gezogen werden\footnote{Vgl. Akz. CP-01/05: Marrkan gg. Hòirran}.

\section{Struktur der Exekutive}
\subsection{Kaisergarde}
a.	Die Kaisergarde ist die Exekutivgewalt des Kaiserreichs auf nationaler Ebene. Sie sind dazu berechtigt, polizeiliche Kontrollen durchzuführen, Personen ohne gerichtlichen Beschluss in Untersuchungshaft unterzubringen, Personen zu verhören und Personen, wenn keine andere Möglichkeit zum Strafvollzug besteht, straffrei zu töten.\\
b.	Der Kaisergarde sitzt der König von Lúinna vor.

\subsection{Kaiserliche Armee}
a. Die Kaiserliche Armee ist die höchste Exekutivgewalt des Kaiserreichs. Ihr sitzt der König von Duumarkng vor.\\
b. Für Itagoren besteht Wehrpflicht\\
c. Entzieht man sich der Wehrpflicht, muss man 100 HTK zahlen

\subsection{Palastgarde}
Die Palastgarde ist eine Spezialeinheit der Armee und dient als Wache für Regierungseinrichtungen. Sie ist unterteilt in die\\
a. Kingsford-Garde\\
b. Mordorianische Garde

\subsection{Polizeiliche Kontrollen}
a. Polizeiliche Kontrollen werden durch Staatspolizisten ausgeführt.\\
b. Sie dürfen Leute, die illegale Dinge mit sich führen, töten sofern diese der dritten Aufforderung, sie abzugeben, nicht nachgehen.\\
c. Auch dürfen sie Leute, die kein gültiges Visum mit sich tragen und dennoch sich nach der dritten Aufforderung noch auf dem Gebiet aufhalten, einsperren.\\
d. Ausnahmen hierzu bilden eingeladene Personen.

\subsection{Strafverfolgung}
a. Entzieht man sich der Strafverfolgung des Reichs, wird man auf dem Gebiet für vogelfrei erklärt, es sei denn, man stellt sich freiwillig vor das Hohe Tribunal.\\
b. Man darf sich ebenfalls nicht der Strafverfolgung verbündeter Reiche auf dem Gebiet des Kaiserreichs entziehen.\\
c. a und b treten nur dann ein, wenn einer Person kein Asyl gewährt wurde.\\
d. Einer Person darf Asyl gewährt werden, wenn sie in einem anderen Staat eine Straftat beging, die auf dem Gebiet des Kaiserreichs nicht als Verbrechen anerkannt wird.\\
e. Das Recht auf Asyl darf einer Person jederzeit entzogen werden\\
f. Behindert man die Justiz absichtlich, so muss man eine Bußgeldstrafe in Höhe von 30 HTK zahlen. 

\section{Strafrecht}
\subsection{Diebstahl}
a. Stiehlt man vom Territorium des Kaiserreichs, so muss man die Ware mitsamt ihres doppelten Warenwerts, sofern vorhanden, zurückerstatten. Andernfalls muss der doppelte Warenwert gemäß üblichem Marktpreis gezahlt werden.\\
b. Dies gilt für alle Gegenstände, die dem Staatsgebiet entstammen oder einer Person auf dem Staatsgebiet gehören und widerrechtlich entwendet wurden.\\
c. Auch gilt dies für Gegenstände, die gelöscht wurden.

\subsection{Mord}
a. Tötet man eine Person vorsätzlich, so muss man die Person mit 1000 HTK entschädigen und wird hingerichtet.
b. Das hamavarische Recht unterscheidet nicht zwischen Mord und Totschlag.

\subsection{Körperverletzung}
Wer eine Person auf dem Gebiet des Kaiserreichs physisch verletzt, muss mit einer Strafe von 15 HTK rechnen.

\subsection{Schwere Körperverletzung}
Verletzt man eine Person vorsätzlich so schwer, dass sie mindestens die Hälfte ihrer Leben verloren hat, so muss man 50 HTK zahlen.

\subsection{Verunglimpfung fraktioneller Insignien und Symbole}
a. Wer fraktionelle Symbole von Hamavar, dessen Vasallen oder Verbündeten verunglimpft oder absichtlich entfernt, muss 50 HTK zahlen.\\
b. Hierzu zählt ebenfalls das unerlaubte Tragen von Orden und Uniformen, beziehungsweise das Tragen von Orden zu einer inoffiziellen Uniform.

\subsection{Effekte und Fähigkeiten}
Man darf keine Effekte ohne Genehmigung haben. Verstöße werden mit 30 HTK Bußgeld vergolten.

\subsection{Verbotene Gegenstände}
Man darf keine verbotenen Gegenstände mit sich führen, ansonsten droht eine Hinrichtung.

\subsection{Pferde}
Pferde sind innerhalb der Stadt nicht als Fortbewegungsmittel gestattet. Jeglicher Verstoß wird mit einer Bußgeldstrafe von 2 HTK geahndet.\\
a. Reitet man mit einem Pferd in den Palasthof des Weißen Palasts, so muss man 6 HTK zahlen.

\subsection{Betrug}
Wer sich oder einen Dritten durch Vorspiegelung falscher Tatsachen bereichern oder einen Vorteil verschaffen möchte, muss Bußgeld zahlen. Der Betrag wird an die Schwere der Straftat angepasst.

\subsection{Sklaverei}
Sklaverei und Menschenhandel werden mit 40 HTK Bußgeld und der Todesstrafe bestraft.

\subsection{Menschenexperimente}
Menschenexperimente sind nur unter staatlicher Aufsicht erlaubt.\\
a. Dies erfordert kein Einverständnis der Testperson.\\
b. Der Staat kann Einspruch gegen die Wahl der Testperson erheben.

\subsection{Geldwäsche}
Wer sich ohne Genehmigung der Lotos-Bank HTK oder Makedonische Drachmen prägt, muss eine Haftstrafe absitzen. Weiterhin wird das Konto der Person geleert und ihr temporär alle Geldzufuhren abgestellt. Die Person verliert somit ihre Kreditfähigkeit und all ihre Immobilien. Alles weitere wird gemäß \ref{vergehen} gehandhabt.

\subsection{Siegelfälschung}
Wer ein Schwarzsiegel, staatliches Zertifikat oder einen historischen Gegenstand ungenehmigt dupliziert, muss 1000 HTK Strafe zahlen. Zudem muss der Gewinn, der dadurch erwirtschaftet wurde, zurückgezahlt werden.

\subsection{Hehlerei}
Wer illegale Waren verkauft, muss 50 HTK Strafe zahlen.\\
a. Gewerbsmäßige Hehlerei wird zusätzlich mit dem Tode vergolten.

\section{Zivilrecht}
\subsection{Sachbeschädigung}
Wer fremdes Eigentum auf dem Gebiet von Hamavar beschädigt, muss für die Schäden vollständig aufkommen und zusätzlich 100 HTK zahlen.

\subsection{Rechte des Eigentümers}
Wer auf hamavarischem Grund rechtmäßig Eigentum erworben hat, darf dieses nutzen und verändern, wie er möchte, solange diese Handlungen ausschließlich gesetzeskonform sind.\\
a. Erwirbt man ein Haus, so gehört einem nur das Innere des Hauses und nicht die Fassade, weshalb diese nicht verändert werden darf.\\
b. Für Territorien gilt, dass man sie erst mit Genehmigung des Lehnsherrn bebauen darf.

\subsection{Schulden}
a. Jegliche Schulden, die man beim Kaiserreich, dem Adel oder den Bürgern des Kaiserreichs hat, müssen innerhalb von 10 Tagen zurückgezahlt werden.\\
b. Tut man dies nicht, verliert man bis zur Rückzahlung zusammen mit zusätzlichen 80 HTK oder Gegenständen mit äquivalentem Wert die Kreditfähigkeit im Kaiserreich.\\
c. Die Strafe nach dreifachem Aufschub liegt im Ermessen des zuständigen Gerichts.

\section{Staatsrecht}

\subsection{Betreten des Staatsgebietes}
Das Betreten des Staatsgebietes darf nur mit einer ausdrücklichen Genehmigung erfolgen. Betritt man das Staatsgebiet ohne diese Aufenthaltsgenehmigung, so muss man 50 HTK Strafe zahlen.

\subsection{Spionage}
Strategische Aufklärung und Spionage auf dem Staatsgebiet sind nicht erlaubt und daher strafbar. Aufgrund der besonderen Schwere wird dies mit einer Hinrichtung und 1000 HTK Strafe vergolten.\\
a. Dies gilt nicht für Operationen, die durch den Staat ausdrücklich genehmigt wurden. \\
b. Man darf ebenso wenig ohne Genehmigung das hamavarische Territorium im Zuschauermodus durchqueren, denn gilt dies ebenfalls als Spionage.

\subsection{Strafen in den Reichsstädten}
In den Reichsstädten gelten die fünffachen Bußgeldsätze.

\subsection{Finanzeller Status des Reichs}
Der Kaiser kann auf nationaler Ebene nicht verschuldet sein.

\subsection{Hochverrat}
a.	Als Hochverräter gilt, wer
\begin{enumerate}[a)]
	\item Staatsgeheimnisse ohne Genehmigung verbreitet oder versucht auf diese unerlaubt zuzugreifen.
	\item Eine absichtliche Schwächung des Staates herbeiführt
	\item Die Befehle des Kaisers verweigert
\end{enumerate}
b.	Der Strafsatz gleicht dem Strafsatz des Mordes an dem Kaiser.

\subsection{Religiöse Gegenstände}
Wer Gegenstände religiöser Natur beschädigt oder zerstört oder auf religiösem Boden Verbrechen begeht, muss eine Bußgeldstrafe in Höhe von 200 HTK zahlen und wird hingerichtet.\\
a. Entweiht man religiöse Gebäude kommt dies dem dreifachen Strafsatz gleich.

\subsection{Majestätsbeleidigung}
Beleidigt man den Kaiser, den Staat oder übergeordnete Staatsvertreter, so muss man 30 HTK Strafe zahlen.

\section{Wirtschaftsrecht}
\subsection{Handelslizenzen}\label{lizenzen}
a. Jede Person, die auf dem hamavarischen Territorium Handel treiben möchte, muss entweder über eine hamavarische Handelslizenz oder eine hamavarische Staatsbürgerschaft verfügen.\\
b. Im Falle eines Unternehmens muss eine hamavarische Handelslizenz beantragt werden, sobald sich auf dem hamavarischen Territorium mindestens eine Person dieses Unternehmens befindet, die die Geschäfte ausführt und über keine hamavarische Staatsbürgerschaft verfügt.\\
c. Als Unternehmen wird jeder Zusammenschluss von Händlern und jede Gesellschaft mit Arbeitnehmern bezeichnet, die mindestens eine Person beschäftigt.

\subsection{Abbaulizenzen}
Wer auf hamavarischem Territorium Ressourcen jedweder Art abbauen und weiterverkaufen möchte, muss eine hamavarische Abbaulizenz beantragen.

\subsection{Gilden}\label{gilden}
Personen, die ein Handwerk ausüben möchten, müssen sich einer, durch den Staate Hamavars anerkannten Gilde anschließen.

\subsection{Handelsbeschränkungen}\label{beschraenkungen}
a. Auf dem hamavarische Staatsgebiet dürfen lediglich Waren gehandelt werden, dessen Abbau, Produktion, sowie Weiterverkauf gemäß \ref{lizenzen} - \ref{gilden} legal sind.\\
b. Wer gegen dieses Gesetz verstößt, muss mit einem Bußgeld in Höhe von mindestens 100 HTK rechnen. Wiederholungstaten werden mit weitaus höheren Bußgeldern bestraft. Ist eine Handelslizenz vorhanden, wird diese ebenfalls entzogen.

\subsection{Verpflichtungen der Unternehmen und Händler}
Händler und Unternehmen sind dazu verpflichtet, ihre Waren mit Kennzeichen zu versehen, die eine Fälschung ausschließen und somit die Nachvollziehbarkeit des Produktionsweges für die Kontrolle der Einhaltung von \ref{beschraenkungen} ermöglichen. Sind diese Kennzeichen nämlich nicht vorhanden, ist es möglich, dass die Kunden wegen Verstoßes gegen \ref{beschraenkungen} zur Verantwortung gezogen werden.

\subsection{Schutz staatlicher Firmen}
a. Staatliche Firmen befinden sich unter dem Schutz des Staates und werden somit als Staatseigentum gehandhabt. Die Strafen werden dementsprechend angepasst.
b. Zu den staatlichen Firmen gehören:
\begin{enumerate}
	\item Die South Hamavarian Company
	\item Die Lotos-Bank
\end{enumerate}

\end{document}