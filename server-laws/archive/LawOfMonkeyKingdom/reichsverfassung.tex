\documentclass{article}
\usepackage[utf8]{inputenc}
\usepackage{textcomp}
\usepackage{amsmath}
\usepackage{enumerate}
\usepackage{ragged2e}
\usepackage{blindtext}

\renewcommand{\thesection}{\Roman{section}}
\counterwithout{subsection}{section}
\renewcommand{\thesubsection}{§\arabic{subsection}}

\title{Codex Imperatoris Imperii Germaniae}
\author{Erik VIII. }
\date{23. Mai 1991}

\begin{document}
\maketitle
\vspace*{\fill}
\paragraph{Die Verfassung des Deutschen Kaiserreichs gemäß Gesetzesentwurf von Reichskanzler Franz-August von Preußen am 14. März 1991, im Namen Seiner Majestät Kaiser Titus von Deutschland.}

\newpage
\topskip0pt
\vspace*{\fill}
\begin{Center}
\textbf{1. Auflage}
\end{Center}
\begin{flushright}
\textit{Reichsstadt Berlin, den 09. April 2023}
\end{flushright}
\textbf{Auf Anordnung Seiner Majestät Kaiser Titus von Deutschland hin, kamen Reichskanzler Franz-August von Preußen umfassende Befähigungen bezüglich der Gesetzgebung zu. Das Fundament dieser Auflage bildet ein durchgesetztes imperiales Edikt vom 14. März 1991. Zur Meidung aufkommender Unannehmlichkeiten für Seine Majestät, wird der Leser gebeten, bei Verfassungsbeschwerden zunächst an den Autoren heranzutreten (auf dem Server als "Oberster Richter" markiert) und bei äußerster Notwendigkeit diese dem Reichsverfassungsgericht vorzulegen.}
\\\\
\textbf{- Reichskanzler König Franz-August von Preußen aus dem Hause von Preußen-Wittelsbach}
\\\\\\\\\\\\\\\\\\\\\\\\\\\\\\\\
\vspace*{\fill}
%
\newpage
\tableofcontents
\newpage
\section{Präambel}
Im Namen des Heiligen Auserwählten, dem Abkömmling der Gottkaiser, Seiner Majestät Kaiser Titus von Deutschland, Praetor Imperatoris der Herzogtümer Böhmen und Mähren, sowie der Königreiche Belgien, Niederlande, Luxemburg, England, Schottland und der Fürstentümer Wales und Nordirland, der Legitimation eines gerechten Staates und der Vernunft dienlich, rufe ich, Reichskanzler König Franz-August von Preußen, Sohn und Abkömmling des Hauses von Preußen, sowie erster der Linie von Preußen-Wittelsbach, die Reichsverfassung von Deutschland, die den Namen Codex Imperatoris trägt, aus. Möge ihr folgeleisten, wer sich dem Kaiser zu unterwerfen weiß. .
Dem Kaiser sei treu, wer Herr seiner Vernunft ist.\\\\
\textit{Ewigkeit dem Kaiser, seinem Kaiserreich und der Allmächtigen, die über uns wachen}
\newpage
\section{Feudalordnung}
\subsection{Feudalstruktur}
(1) Dem Staate unterstehen jegliches Territorium auf dessen Gebiet und die Einwohner, sowie Besucher jenes Gebietes.\\
(2) Die Pflicht jedes Menschen auf dem Territorium des Staats ist es, den Befehlen des Kaisers ausnahmslos folgezuleisten.\\

\subsection{Der Kaiser}
(1) Der Kaiser ist das Staatsoberhaupt des Kaiserreichs von Deutschland und daher in der Hierarchie an oberster Stelle. \\ 
(2) Jedes Mitglied der Nation ist ihm zu Treue verpflichtet.\\
(3) Er regiert das Kaiserreich und verfügt daher über absolute Entscheidungsvollmachten.\\
(4) Der Kaiser steht über dem Gesetz.\\
(5) Der Kaiser verfügt sowohl über das Besitz- als auch Verwaltungsrecht seiner Domänen.\\
(6) Die direkte Anrede lautet "Eure Majestät".\\
(7) Die indirekte Anrede lautet "Seine Majestät".\\
(8) Im rechtlichen Kontext ist der Begriff des Kaisers synonym mit dem Begriff des Staats, des Kaiserreichs von Deutschland und dessen weiterer Synonyme.\\
(9) Der Reichskanzler ist der Stellvertreter des Kaisers.

\subsection{Das Ministerkabinett}
Als Ministerkabinett wird die Versammlung bezeichnet, die aus dem Kaiser und den Ministern (\ref{koenige} Abs. 6) besteht. Sie fungiert nicht als Parlament, da die Ratsabstimmungen lediglich beratender Funktion sind und daher nicht vom Kaiser berücksichtigt werden müssen.

\subsection{Der Hochadel}
Dem Hochadel gehören der gesamte Land- und Verwaltungsadel der Herzogs-, Königreichs- und Kaiserreichsebene an.

\subsection{Die Minister}\label{koenige} 
(1) Als Minister wird bezeichnet, wer von dem Kaiser oder dem Reichskanzler als Mitglied des Ministerkabinetts eingesetzt wurde.\\
(2) Der Titel bleibt bis zur nächsten Legislatur im eigenen Besitz.\\
(3) Jeder Minister fungiert als Berater in seinem Geschäftsbereich.\\
(4) Die Geschäftsbereiche lauten:
\begin{enumerate}
	\item Der Reichsaußenminister
	\item Der Reichskriegsminister
	\item Der Reichswirtschaftsminister
\end{enumerate}

\subsection{Der Reichskanzler}
(1) Der Reichskanzler ist der Regierungschef und sitzt dem Ministerkabinett vor.\\
(2) Da er Stellvertreter des Kaisers ist, ist sein Wort in Abwesenheit des Kaisers absolut und kann nur durch den Kaiser selbst aufgehoben werden.\\
(3) Der Titel ist lebenslang und wird an den Ertgeborenen des Hauses von Preußen verliehen.

\subsection{Vasallen}
Vasallen dürfen ihre Vasallen, sofern sie das Recht auf Vasallen innehaben, selbst wählen. Sie müssen ihrem Lehnsherrn Treue und Tribut leisten, während dieser den Vasallen Schutz bieten muss. Ebenfalls müssen sie bei Entscheidungen, die ihre Vollmachten übersteigen, ihren Lehnherrn fragen.

\subsection{Die Könige}
(1) Einen Königstitel erhält, wer aus historischen Gründen den Königstitel vom Kaiser zuerkanntbekommen hat.\\
(2) Der König steht auf einer Ebene mit den Herzögen und Großherzögen, verfügt allerdings über mehr Vollmachten als diese.\\
(3) Die Könige unterstehen dem Kaiser.\\
(4) Könige müssen direkt mit "Eure Königliche Hoheit" angesprochen werden.\\
(5) Indirekt müssen diese mit "Seine/Ihre Königliche Hoheit" adressiert werden.

\subsection{Die Herzöge}
(1) Herzöge stehen auf der gleichen Ebene, wie Könige, bekommen allerdings weniger Vollmachten zuerkannt.\\
(2) Herzöge müssen mit "Euer Gnaden" angesprochen werden.\\
(3) Die indirekte Anrede lautet "Seine/Ihre Gnaden".\\
(4) Herzöge unterstehen dem Kaiser.\\
(5) Bei dem Großherzog handelt es sich um eine Sonderform, die allerdings ebenfalls keine gesonderten Rechtsansprüche geltend machen können.

\subsection{Die Fürsten}
(1) Fürsten unterstehen Herzögen. Großherzögen und Königen.
(2) Sie verfügen über keine gesonderte Anrede.

\subsection{Autonome Gebiete}
(1) Gebiete, die vom Kaiser den Status eines autonomen Gebiets zuerkannt bekommen haben, dürfen eine eigene Regierung bilden und die nationale Politik ohne Einflussnahme des Kaisers führen.\\
(2) Ihnen stehen militärische Entscheidungen jedoch nicht frei.\\
(3) Sie müssen sich ebenfalls als Gebiet unter der deutschen Kaiserkrone zu erkennen geben.

\subsection{Sonderrechte}
Dem Kaiser ist es gestattet, durch die Anwendung des Lex Votum Imperatoris jegliche Entscheidung jeglicher Instanz aufzuheben.

\subsection{Die Herrschaft des Kaisers}
(1) Der Kaiser regiert uneingeschränkt bis zu seinem Lebensende und wählt vor seinem Tod einen Nachfolger aus seiner Linie.\\
(2) Die Autorität des Kaiser darf nicht angezweifelt werden.  \\
(3) Der Kaiser kann Ausnahmen zu allen Gesetzen erlassen.  \\

\subsection{Staatsbürger}
Als Staatsbürger werden die Einwohner vom Kaiserreich Deutschland bezeichnet, denen eine Staatsbürgerschaft gewährt wurde.

\subsection{Banner}
(1) Jeder Feudalherr, als welche da gelten:
\begin{enumerate}
	\item Die Könige
	\item Die Herzöge
	\item Die Großherzöge
	\item Die Fürsten
\end{enumerate}
sind verpflichtet, ein eigenes Banner zu führen.\\
(2) Die Banner dürfen weder anstößige, noch auf sonstige Weise auf diesem Server, dem Discord-Server oder gemäß Recht der Bundesrepublik Deutschland verbotene Symbolik aufweisen.\\
(3) Das Banner muss über eine angemessene Komplexität verfügen.

\section{Struktur der Judikative}

\subsection{Gerichtliche Instanzen}
(1) Erhebt eine Partei Anklage, so beginnt der Rechtsstreit in der untersten Instanz. Sofern man gemäß \ref{verlauf} Nr. 10 in Berufung gegangen ist, wird das Verfahren von der nächsten Instanz behandelt.\\
(2) Die Instanzen in aufsteigender Folge sind:
\begin{enumerate}
	\item Kammergericht
	\item Landesgericht
	\item Reichsgerichtshof
	\item Reichsverfassungsgericht
\end{enumerate}
(3) Der Kaiser kann nach eigenem Ermessen den Richterschaftsvorsitz eines Prozesses jederzeit übernehmen.

\subsection{Kammergericht}
Die Kammergericht ist für Rechtsstreitigkeiten auf Fürstentumsebene zuständig. Ihr sitzt der jeweilige Fürst vor.

\subsection{Landesgericht}
Dem Landesgericht sitzt der jeweilige Herzog, beziehungsweise König vor. Dementsprechend ist sie auf der Herzogtumsebene tätig.

\subsection{Reichsgerichtshof}
(1) Der Reichsgerichtshof ist das oberste Gericht des Kaiserreichs Deutschland und besteht aus dem Ministerkabinett. Sein Vorsitzender ist der Reichskanzler.\\
(2) Dieses Gericht agiert auf Reichsebene und ist somit für jede Prozesse der Reichsgerichtsbarkeit verantwortlich.

\subsection{Reichsverfassungsgericht}
Das Reichsverfassungsgericht ist ein Gericht, welches separat zu den herkömmlichen Instanzen zu sehen ist. Es befindet sich in der Lage, Verfassungsänderungen vorzunehmen und Verfassungsbeschwerden zu behandeln. Im Gegensatz zu den vorherigen Instanzen kann dieses Gericht die Entscheidungen der vorangehenden Gerichte nur bei einem eindeutig vorliegenden Widerspruch zum Codex Imperatoris aufheben. Die Richterschaft besteht aus dem Reichskanzler als der Vorsitzende, sowie zwei, von ihm gewählten, Richtern beratender Funktion.

\subsection{Zeugen}\label{zeugen}
Man darf Personen in den Zeugenstand berufen.\\
(1) Diese darf man unter den gegebenen Regeln befragen  \\
(2) Diese Regeln lauten:  \\
\begin{enumerate}
	\item Die Zeugen stehen automatisch unter Eid, sobald sie ihr erstes Wort im Zeugenstand erheben.
	\item Die Zeugen müssen daher alles wahrheitsgemäß beantworten.
	\item Jegliche ungenauen Aussagen der Zeugen werden nicht ins Protokoll aufgenommen (siehe hierzu \ref{eordnung} Abs. 6).
\end{enumerate}

\subsection{Anwälte}
Man darf einen Anwalt einstellen. Hierbei muss jedoch beachtet werden, dass kein Anrecht auf einen Pflichtverteidiger besteht.

\subsection{Einspruchsordnung}\label{eordnung}
(1) Einsprüche sind erlaubt und bilden eine Ausnahme zu \ref{gordnung} Abs. 1.\\
(2) Sie können durch die Richterschaft abgewiesen werden.\\
(3) Bei einmaliger Ablehnung eines Einspruchs darf dieser nicht auf dieselbe Aussage erneut angewandt werden.\\
(4) Auf die Ankündigung eines Einspruchs muss stets die Ankündigung des Grundes folgen.\\
(5) Rechtlich zulässige Gründe sind:
\begin{enumerate}
\item Nicht aussagekräftig/unverständlich/mehrdeutig: Die Aussage oder Frage ist aufgrund seiner nicht aussagekräftigen Natur unzulässig.
\item Bereits beantwortet: Die gleiche Frage wurde mehrfach gestellt, obwohl sie bereits beantwortet wurde.
\item Unbewiesene Vermutung: Der Befragende behauptet etwas, ohne sich auf vorliegende Beweise zu stützen.
\item Fordert Spekulationen: Der Befragende fordert den Zeugen auf, zu spekulieren.
\item Supra interrogatio (über Befragung hinaus): Der Anwalt fragt mehr als eine Frage gleichzeitig.
\item Mangelnde Kenntnisse: Die Kenntnisse des Zeugens über das gefragte Thema sind unzureichend nachgewiesen.
\item Ohne Priorität: Die Frage ist dem Prozess beziehungsweise der Befragung nicht dienlich.
\item Gerücht: Die Antwort der Partei baut auf außergerichtlichen Aussagen auf.
\item Hinterfragt die Staatsautorität: Eine Partei fechtet, hinterfragt oder beleidigt die Staatsautorität beziehungsweise die Autorität des Kaisers. Wird dieser Einspruch bewilligt, wird derjenige, der die Aussage gebracht hat, hinterher wegen Verstoßes gegen \ref{verrat} vor Gericht gestellt.
\end{enumerate}
(6) Wird ein Einspruch stattgegeben, so muss der Befragende bei der Befragung mit der nächsten Frage fortfahren. Der Zeuge darf die vorherige Frage nicht beantworten oder seine Aussage wird im Fall, dass er sie bereits getätigt hat oder dennoch antwortet, gestrichen. Erhebt ein Richter diesen Einspruch, so ist dem sofort stattgegeben, sofern der Gerichtsvorsitzende dem nicht widerspricht.

\subsection{Prozessverlauf}\label{verlauf}
Das Recht des Kaiserreichs Deutschland sieht den nachfolgenden Verlauf für Gerichtsverfahren vor.\\
\begin{enumerate}
	\item Alle Parteien mit Ausnahme der Richterschaft betreten den Raum.
	\item Die Richterschaft versammelt sich. Währenddessen muss jeder Anwesende stehen.
	\item Der Gerichtsvorsitzende eröffnet den Prozess und die weiteren Richter setzen sich.
	\item Der Gerichtsvorsitzende verliest die Anklageschrift.
	\item Der Kläger muss den Strafbestand aus seiner Sicht darlegen.
	\item Der Beklagte hat das Wort und darf seine Darstellung des Sachverhalts darlegen.
	\item Von nun an entscheidet der Gerichtsvorsitzende, wer das Wort erhält.
	\item Sobald alle Beweise und Aussagen der beiden Parteien dargelegt wurden, dürfen die beklagte Partei und die klagende Partei, beziehungsweise deren Vertreter, je ein Strafmaß, beziehungsweise den Freispruch, empfehlen.
	\item Die Richterschaft tritt zurück und berät sich in einem separaten Gespräch. Hierbei wird über die Strafe beratschlagt und anschließend entschieden. Bei Stimmgleichheit verfügt der Gerichtsvorsitzende eine zweite Stimme.
	\item Die Richterschaft betritt den Saal, wobei erneut jeder stehen muss, und verkündet im Anschluss die Strafe. Daraufhin fragt der Gerichtsvorsitzende, ob eine Partei in Berufung gehen möchte, sofern denn eine höhere Instanz besteht. Andernfalls ist die Strafe final.
	\item Bis der letzte Richter den Saal verlassen hat müssen alle Teilnehmer stehen und dürfen den Saal nicht verlassen.
\end{enumerate}

\subsection{Gerichtsordnung}\label{gordnung}
(1) Man darf nicht unaufgefordert sprechen\\
(2) Verstöße gegen die Gerichtsordnung unter Inbezugnahme von \ref{zeugen} und \ref{verlauf} werden, sofern bereits eine Verwarnung erteilt wurde mit 10 HTK Bußgeld geahndet. Liegen nach Ermessen der Richterschaft zu viele Verstöße vor, können sie die schuldige Partei ungeachtet ihrer Relevanz für diesen Prozess aus dem Saal verweisen und das Verfahren anschließend in dessen Abwesenheit fortfahren.\\
(3) Von Absatz 2 ist lediglich der Kaiser ausgenommen.

\subsection{Gerichtliche Vorladung}
Sofern ein Verfahren bestätigt wurde kann unter Vereinbarung mit beiden Parteien ein Gerichtstermin festgelegt werden. Dies wird als außerordentliche Vorladung angesehen.\\
(1) Legt das Gericht einen Termin fest, so muss dieses beide Parteien in einem Schreiben deutlich über das Verhandlungsdatum informieren. Hierbei handelt es sich um eine ordentliche Vorladung\\
(2) Der Termin und Ort einer Verhandlung muss spätestens zwölf Stunden vor Prozessbeginn bekanntgegeben werden.\\
(3) Ein Antrag auf Aufschub kann bis zu zwei Stunden vor Prozessbeginn eingereicht werden.\\
(4) Wird diesem Antrag durch den Gerichtsvorsitzenden des Verfahrens stattgegeben, so wird das Verfahren vertagt.\\
(5) Andernfalls, oder wenn kein Antrag besteht, müssen die Parteien erscheinen, ansonsten wird in ihrer Abwesenheit verhandelt.\\
(6) Erscheint keine Partei, so wird der Termin ebenfalls vertagt.\\
(7) Jeder gemäß Absatz 5 abwesenden Partei droht eine Bußgeldstrafe in Höhe von 20 HTK.\\

\subsection{Anrede des Richters}
Steht man vor Gericht, so hat man den Richter mit der, ihm zustehenden Adressierung anzureden. Tut man dies nicht, wird gemäß \ref{gordnung} verfahren.\\

\subsection{Rechtliche Immunität}
(1) Mangelnde oder fehlerhafte Kenntnisse des Gesetzes gewähren keine rechtliche Immunität, da das Informieren über die Gesetzeslage Pflicht ist.\\
(2) Der Kaiser darf Personen rechtliche Immunität verleihen.

\subsection{Vergehen am Hochadel}\label{vergehen}
(1) Vergehen an dem Hochadel werden mit dem dreifachen Strafsatz vergolten.\\
(2) Vergehen an dem Kaiser werden mit dem zehnfachen Strafsatz vergolten.\\
(3) Vergehen an dem Staat gelten als Vergehen an dem Kaiser.

\subsection{Bußgeldstrafe}
Bußgeldstrafen sind Strafen, die die Schadensvergütung im Gegenständlichen, wie auch Geistigen und Symbolischen, in finanzieller Form anstreben.\\
(1) Sie werden bei Straftaten minderer Schwere als umfassendes Strafmaß erhoben.\\
(2) Bei Straftaten besonderer Schwere sind sie als zusätzliche Strafe angeführt.\\
(3) Die Bußgeldstrafen werden gemäß aktuellem Wechselkurs des Hamavarischen Torrekhen (HTK) in Coins errechnet.\\
(4) Bußgeldstrafen können auch durch Waren äquivalenten Werts ersetzt werden. Hierbei muss die Richterschaft jedoch die Waren als angemessen betrachten, andernfalls müssen andere Waren angeboten werden.

\subsection{Freiheitsstrafe}
(1) Eine Haftstrafe kann bei Beschluss des Gerichts entweder als Strafersatz oder Strafzusatz angewendet werden.\\
(2) Bei Ausbruchsversuchen und Ausbrüchen werden stets zehn Minuten zusätzliche Haft angeordnet.\\
(3) Beihilfe bei Ausbrüchen werden mit dem Verordnen der gleichen Haftstrafe für die helfende Partei bestraft.\\
(4) Abgesessen hat man die Strafe, sobald man die jeweilige Zeit nachweislich online war.\\
(5) Der Staat haftet für keine Gegenstände, die während der Haftstrafe verlorengehen, sofern für den Häftling genügend Zeit bestand, die Gegenstände anderweitig zu lagern.\\
(6) Der Strafsatz bemisst sich in 5-Minuten-Sätzen

\subsection{Hinrichtung}
(1) Hinrichtungen sind als Strafmaßnahme für Kapitalverbrechen vorgesehen.\\
(2) Hinrichtungen sind erst dann erlaubt, wenn das Gericht eindeutig eine Hinrichtung verhängt hat.\\
(3) Diese Strafe kann nur auf Empfehlung des Klägers hin aufgehoben werden\\
(4) Im Falle eines Verstoßes gegen das Staatsrecht verliert Absatz 3 seine Wirksamkeit.

\subsection{Verbindlichkeit von Strafsätzen}
(1) Die aufgeführten Strafsätze dienen lediglich zur Orientierung und sind daher nicht verpflichtend.\\
(2) Dies gilt nicht für Hinrichtungen.\\
(3) Bei Wiederholungstaten liegt es je nach Häufigkeit und Schwere der Tat im Ermessen des zuständigen Gerichts, ob weiterhin derselbe oder ein verhärteter Strafsatz geltend gemacht werden sollte.\\
(4) Bei äußerster Häufigkeit oder relativer Häufigkeit von Taten besonderer Schwere, haben Wiederholungstaten die Todesstrafe zur Folge.

\subsection{Untersuchungshaft}
Besteht die Gefahr, dass ein Tatverdächtiger bis zu seinem Prozess flieht oder befragt werden muss, muss eine Unterbringung in der Untersuchungshaft angeordnet werden.\\

\subsection{Unterbringung in Hochsicherheitseinrichtungen}
(1)	Freiheitsstrafen in Höhe von mehr als zwanzig Minuten müssen in Hochsicherheitseinrichtungen abgesessen werden.\\
(2)	Besteht eine akute Fluchtgefahr, so kann dies auch bei kürzerer Haft angeordnet werden.

\subsection{Unterbringung in einer Sonderverwahrung}
Personen, die sich eines Kapitalverbrechens schuldig gemacht haben und daher hingerichtet werden sollen, müssen sofern zusätzlich eine Freiheitsstrafe angeordnet wurde, in einer Todeszelle untergebracht werden. Mit Ende ihrer Haftstrafe werden sie hingerichtet.

\subsection{Entzug von Titeln}
Es ist dem Reichsgerichtshof gestattet, bestimmten Personen den Titel zu entziehen, sofern sie dessen Macht missbrauchen oder mit ihr anderweitig nicht umgehen können.

\subsection{Präzedenzfälle} \label{praez}
Sofern ein rechtlicher Ausnahmefall vorliegt, ist der Fall unter sofortiger Wirkung dem Reichsverfassungsgericht zu übertragen.\\
(1) Entscheidet dieses, dass es sich bei dem vorliegenden Fall um eine Straftat handelt, so muss dies umgehend in die Gesetze aufgenommen werden und
sofern nach Ermessen des Reichsverfassungsgerichts ein Bewusstsein des Verstoßes gegen moralische Normen durch die Beklagte vorliegen sollte, der Strafe entsprechend
geurteilt werden.\\
(2) Absatz 1 darf nicht aus dem alleinigen Grund, einen Schuldspruch zu erzwingen, angewendet werden, sondern dient lediglich der Abwendung absichtlichen Missbrauchs von Gesetzeslücken um Straftaten besonderer Schwere zu verüben.

\subsection{Generationenrecht}
(1) Verstirbt ein Kläger oder Opfer eines Verbrechens, so darf das Haus des Geschädigten Anklage erheben oder die Geschädigte vor Gericht vertreten.\\
(2) Verstirbt ein Täter, so muss sich das Haus des Täters für dessen Straftaten verantworten.\\
(3) Das Haus wird stets durch dessen Oberhaupt vertreten. Besteht keines, so wird dieses vom zuständigen Gericht gewählt.\\
(4) Gemäß Absatz 2 können demnach auch die nachfolgenden Oberhäupter zur Rechenschaft gezogen werden.

\subsection{Strafverfolgung}
(1) Entzieht man sich der Strafverfolgung des Reichs, wird man auf dem Gebiet für vogelfrei erklärt, es sei denn, man stellt sich freiwillig vor das zuständige Gericht.\\
(2) Man darf sich ebenfalls nicht der Strafverfolgung verbündeter Reiche auf dem Gebiet des Kaiserreichs entziehen.\\
(3) Absatz 1 und 2 treten nur dann ein, wenn einer Person kein Asyl gewährt wurde.\\
(4) Einer Person darf Asyl gewährt werden, wenn sie in einem anderen Staat eine Straftat beging, die auf dem Gebiet des Kaiserreichs nicht als Verbrechen anerkannt wird.\\
(5) Das Recht auf Asyl darf einer Person jederzeit entzogen werden\\
(6) Behindert man die Justiz absichtlich, so muss man eine Bußgeldstrafe in Höhe von 30 HTK zahlen.\\
(7) Nicht vollständig ausgezahlte Bußgelder werden gemäß \ref{schulden} gehandhabt.

\subsection{Prozessbezeichnung}
(1) Jeder Prozess verfügt über eine einzigartige Bezeichnung, die die folgenden Komponenten enthält:
\begin{enumerate}
	\item Das Aktenzeichen
	\item Die Prozessparteien
\end{enumerate}
(2) Das Aktenzeichen wird wie folgt notiert: Akz. ABK-YY/PN.\\
(3) Folgende Vergehensbezeichnungen, abgekürzt ABK, bestehen nach deutscher Reichsverfassung:
\begin{enumerate}
	\item CP (Casus poenae): Strafrechtlicher Prozess
	\item CC (Casus civilis): Zivilrechtlicher Prozess
	\item RP (Res publica): Staats- oder verwaltungsrechtlicher Prozess
\end{enumerate}
(4) Die Prozessnummer (PN) bezeichnet die Nummer des Prozesses im jeweiligen Jahr (YY).\\
(5) Die Prozessparteien werden wie folgt notiert: "Klagende gg. Beklagte".\\
(6) Ist der Staat der Kläger, so wird als Klagende "Das Kaiserreich Deutschland" notiert.

\section{Strafrecht}
\subsection{Diebstahl}
(1) Stiehlt man vom oder auf dem Territorium des Kaiserreichs, so muss man die Ware mitsamt ihres doppelten Warenwerts, sofern vorhanden, zurückerstatten. Andernfalls muss der doppelte Warenwert gemäß üblichem Marktpreis gezahlt werden.\\
(2) Dies gilt für alle Gegenstände, die dem Staatsgebiet entstammen oder einer Person auf dem Staatsgebiet gehören und widerrechtlich entwendet wurden.\\
(3) Auch gilt dies für Gegenstände, die gelöscht wurden.

\subsection{Mord}
(1) Tötet man eine Person vorsätzlich, so muss man die Person mit 1000 HTK entschädigen und wird hingerichtet.
(2) Dieses Recht unterscheidet nicht zwischen Mord und Totschlag.

\subsection{Körperverletzung}
Wer eine Person auf dem Gebiet des Kaiserreichs physisch verletzt, muss mit einer Strafe von 15 HTK rechnen.

\subsection{Schwere Körperverletzung}
Verletzt man eine Person vorsätzlich so schwer, dass sie mindestens die Hälfte ihrer Leben verloren hat, so muss man 50 HTK zahlen.

\subsection{Verunglimpfung fraktioneller Insignien und Symbole}
(1) Wer fraktionelle Symbole des deutschen Kaiserreichs, dessen Vasallen oder Verbündeten verunglimpft oder absichtlich entfernt, muss 50 HTK zahlen.\\
(2) Hierzu zählt ebenfalls das unerlaubte Tragen von Orden und Uniformen, beziehungsweise das Tragen von Orden zu einer inoffiziellen Uniform.

\subsection{Effekte und Fähigkeiten}
Man darf keine Effekte ohne Genehmigung haben. Verstöße werden mit 30 HTK Bußgeld vergolten.

\subsection{Verbotene Gegenstände}
Man darf keine verbotenen Gegenstände mit sich führen, ansonsten droht eine Hinrichtung.

\subsection{Betrug}
Wer sich oder einen Dritten durch Vorspiegelung falscher Tatsachen bereichern oder einen Vorteil verschaffen möchte, muss Bußgeld zahlen. Der Betrag wird an die Schwere der Straftat angepasst.

\subsection{Menschenexperimente}
Menschenexperimente sind nur unter staatlicher Aufsicht erlaubt.\\
(1) Dies erfordert kein Einverständnis der Testperson.\\
(2) Der Staat kann Einspruch gegen die Wahl der Testperson erheben und somit die Entscheidung annullieren.

\subsection{Geldwäsche}
Wer sich ohne Genehmigung der Kaiserlichen Reichsbank Coins prägt, muss eine Haftstrafe absitzen. Weiterhin wird das Konto der Person geleert und ihr temporär alle Geldzufuhren abgestellt. Die Person verliert somit ihre Kreditfähigkeit und all ihre Immobilien. Alles weitere wird gemäß \ref{vergehen} gehandhabt.

\subsection{Siegelfälschung}
Wer ein Schwarzsiegel, staatliches Zertifikat oder einen historischen Gegenstand ungenehmigt dupliziert, muss 1000 HTK Strafe zahlen. Zudem muss der Gewinn, der dadurch erwirtschaftet wurde, zurückgezahlt werden.

\subsection{Hehlerei}
Wer illegale Waren verkauft, muss 50 HTK Strafe zahlen.\\
(1) Gewerbsmäßige Hehlerei wird zusätzlich mit dem Tode vergolten.

\section{Zivilrecht}
\subsection{Sachbeschädigung}
Wer fremdes Eigentum auf dem Gebiet des Kaiserreichs beschädigt, muss für die Schäden vollständig aufkommen und zusätzlich 100 HTK zahlen.

\subsection{Rechte des Eigentümers}
Wer auf staatlichem Grund rechtmäßig Eigentum erworben hat, darf dieses nutzen und verändern, wie er möchte, solange diese Handlungen ausschließlich gesetzeskonform sind.\\
(1) Erwirbt man ein Haus, so gehört einem nur das Innere des Hauses und nicht die Fassade, weshalb diese nicht verändert werden darf.\\
(2) Für Territorien gilt, dass man sie erst mit Genehmigung des Lehnsherrn bebauen darf.

\subsection{Schulden}\label{schulden}
(1) Jegliche Schulden, die man beim Kaiserreich, dem Adel oder den Bürgern des Kaiserreichs hat, müssen innerhalb von 10 Tagen zurückgezahlt werden.\\
(2) Tut man dies nicht, verliert man bis zur Rückzahlung zusammen mit zusätzlichen 80 HTK oder Gegenständen mit äquivalentem Wert die Kreditfähigkeit im Kaiserreich.\\
(3) Die Strafe nach dreifachem Aufschub liegt im Ermessen des zuständigen Gerichts.

\subsection{Steuerhinterziehung} \label{steuern}
Wer Steuern nicht vorschriftsgemäß bezahlt oder sich derer ordnungsgemäßen Kontrolle entzieht, muss die Steuern in Form von
Bußgeldstrafe mit zusätzlichen 100 HTK entrichten.

\section{Staatsrecht}

\subsection{Betreten des Staatsgebietes}
Das Betreten des Staatsgebietes darf nur mit einer ausdrücklichen Genehmigung erfolgen. Betritt man das Staatsgebiet ohne diese Aufenthaltsgenehmigung, so muss man 50 HTK Strafe zahlen.

\subsection{Spionage}
Strategische Aufklärung und Spionage auf dem Staatsgebiet sind nicht erlaubt und daher strafbar. Aufgrund der besonderen Schwere wird dies mit einer Hinrichtung und 1000 HTK Strafe vergolten.\\
(1) Dies gilt nicht für Operationen, die durch den Staat ausdrücklich genehmigt wurden. \\
(2) Man darf ebenso wenig ohne Genehmigung das deutsche Territorium im Zuschauermodus durchqueren, denn gilt dies ebenfalls als Spionage.

\subsection{Hochverrat}\label{verrat}
(1)	Als Hochverräter gilt, wer
\begin{enumerate}
	\item Staatsgeheimnisse ohne Genehmigung verbreitet oder versucht auf diese unerlaubt zuzugreifen.
	\item Eine absichtliche Schwächung des Staates herbeiführt
	\item Die Befehle des Kaiser verweigert
\end{enumerate}
(2)	Der Strafsatz gleicht dem Strafsatz des Mordes an dem Kaiser.

\subsection{Religiöse Gegenstände}
Wer Gegenstände religiöser Natur beschädigt oder zerstört oder auf religiösem Boden Verbrechen begeht, muss eine Bußgeldstrafe in Höhe von 200 HTK zahlen und wird hingerichtet.\\
(1) Entweiht man religiöse Gebäude kommt dies dem dreifachen Strafsatz gleich.

\subsection{Majestätsbeleidigung}
Beleidigt man den Kaiser, den Staat oder übergeordnete Staatsvertreter, so muss man 1000 HTK Strafe zahlen und wird anschließend hingerichtet.

\subsection{Blasphemie}
Wer die Existenz des Pantheons der Allmächtigen leugnet, deren Anhängerschaft, Andenken oder Weltbild verunglimpft, oder dem Wort derer, ausgetragen durch den allgerechten Kaiser widerspricht, hat 1000 HTK zu zahlen und wird anschließend hingerichtet.

\section{Wirtschaftsrecht}

\subsection{Rechtsgeschäft} \label{rechtsg}
(1) Als gültiges Rechtsgeschäft gilt jeder der nachfolgenden Rechtsakte, sofern dieser gänzlich gesetzeskonform ist:
\begin{enumerate}
	\item Testamentarische Verfügung
	\item Vertragsgeschäfte
\end{enumerate}
(2) Ein Rechtsgeschäft verliert im Kontext von \ref{praez} seine Gültigkeit auch dann, wenn nach gerichtlichem Urteil kein Bewusstsein der Schuld vorliegt.\\
(3) Es bedarf eine Beglaubigung durch einen Notar, der vom Deutschen Reich ernannt wurde, um vor Gericht gültig zu sein.

\subsection{Firmenführung} \label{firm}
(1) Wer eine Firma gründet, muss diese in das Deutsche Reichshandelsregister (\ref{register}) eintragen lassen.\\
(2) Diese Firmen müssen präzise Buch führen (\ref{buch}).\\
(3) Eine Person kann erst dann rechtskräftig zum Eigentümer einer Gesellschaft ernannt werden, sofern ein Rechtsgeschäft (\ref{rechtsg}) vorliegt, in welchem der vorherige Eigentümer die Gesellschaft dem neuen Eigentümer nachweislich überträgt und der neue Eigentümer in das Deutsche Reichshandelsregister (\ref{register}) eingetragen wurde.\\
(4) Sobald ein Eigentümer beabsichtigt, zurückzutreten und kein neuer Eigentümer gemäß Absatz 3 nachfolgt, ist der Eigentümer für die offenen Geschäfte des Unternehmens verantwortlich. Laufen diese aus, so darf dieser zurücktreten.\\
(5) Solange kein Eigentümer gemäß Absatz 3 nachfolgt, darf die Gesellschaft keine neuen Geschäfte aufnehmen.\\
(6) Absatz 3 ff. gilt nur dann, wenn es sich um kein Familienunternehmen (\ref{familien}) handelt.

\subsection{Handelsregister} \label{register}
(1) Einträge im Handelsregister werden vom Deutschen Reichswirtschaftsministerium vorgenommen.\\
(2) Jegliche Gesellschaft muss mit dem Namen des Eigentümers, der Gesellschaftsform gemäß \ref{gesellform}, sowie dem Namen und der Marke der Gesellschaft in das Handelsregister eingetragen werden.

\subsection{Rechtsform} \label{gesellform}
(1) Die Rechtsform einer Gesellschaft bestimmt die Haftungs-, sowie Handelsbedingungen.\\
(2) Im Deutschen Reich anerkannte Rechtsformen sind:
\begin{enumerate}
	\item Handelsgesellschaft logistischer Tätigkeit (HGL)
	\item Handelsgesellschaft deutschen Rechts (HGdR)
	\item Reichsgesellschaft (RG)
	\item Privatunternehmen (PU)
\end{enumerate}
(3) Gesellschaften haften mit dem Kapital der juristischen Person.\\
(4) Ebenso gehört der Umsatz der Gesellschaft der juristischen Person.\\
(5) Die juristische Person von Handelsgesellschaften ist die Handelsgesellschaft selbst.\\
(6) Eigentümer von Privatunternehmen sind dessen juristische Person.\\
(7) Die juristische Person von einer Reichsgesellschaft ist der Staat.\\
(8) Im Falle von Absatz 5 müssen die Eigentümer im Handelsregister den Anteil am Gewinn der Gesellschaft registrieren.\\

\subsection{Auslandshandelsgebühren}
(1) Gesellschaften, die im Ausland Tochterunternehmen eröffnen, müssen zusätzliche Gebühren an das Kaiserreich Deutschland zahlen.\\
(2) Diese Gebühren müssen den Empfehlungen des Reichswirtschaftsministeriums entsprechen.\\ 
(3) Sind die Gebühren zur Eröffnung der Zweigstelle im Ausland billiger, so muss die Gesellschaft die Kostendifferenz zur Empfehlung (Absatz 2) an das Deutsche Reich zahlen.\\
(4) Andernfalls muss es keine Zusatzgebühren erstatten.\\
(5) Von dieser Regelung ausgenommen sind Handelsgesellschaften logistischer Tätigkeit. Im Falle von Absatz 3 müssen sie keine Zusatzgebühren bezahlen und im Falle von Absatz 4 übernimmt das Kaiserreich Deutschland die Kostendifferenz zur Empfehlung.\\
(6) Letzteres verliert seine Wirksamkeit, sofern das Reichswirtschaftsministerium die Unterstützungen innerhalb des fraglichen Staates verwehrt.

\subsection{Familienunternehmen} \label{familien}
(1) Familienunternehmen dürfen nur von deutschen Staatsbürgern gegründet werden.\\
(2) Sie dürfen lediglich von Familienmitgliedern geführt werden.\\
(3) Absatz 2 verliert seine Wirksamkeit, sobald das fragliche Mitglied kein deutscher Staatsbürger ist.\\
(4) Die Gesellschaft kann nur dann an andere Familien ausgehändigt werden, sofern der Eigentümer verfügt, dass die Gesellschaft zu einer nicht-familiären Gesellschaft umgewidmet und anschließend an den außerfamiliären Eigentümer übertragen wird.\\
(5) Im Falle von Absatz 4 kann das Unternehmen nicht zu Lebzeiten des neuen Eigentümers zu einem Familienunternehmen umgewidmet werden.\\
(6) Verstirbt das letzte Mitglied der Familie, so verliert das Familienunternehmen seine Geschäftsfähigkeit gemäß \ref{firm} Abs. 5, sofern keine rechtsgültige Nachfolge bewirkt wurde.

\subsection{Buchführung} \label{buch}
(1) Eine Gesellschaft ist verpflichtet, ohne Auslassungen Buch zu führen.\\
(2) Zu jedem Geschäft muss folgendes vermerkt werden:
\begin{enumerate}
	\item Verkäufer (sofern er von der Hauptgesellschaft abweicht)
	\item Kunde (sofern er von der Hauptgesellschaft abweicht)
	\item Gesamtpreis (dies schließt auch Tauschwaren ein)
	\item Gehandelte Gegenstände, beziehungsweise Kommentar zu gegenstandslosen Transaktionen
\end{enumerate}
(3) Liegt eine gegenstandlose Transaktion, beispielsweise Schenkung oder Spenden in finanzieller oder gegenständlicher Form  vor, so müssen die Kommentare sinnig sein und für das Reichswirtschaftsministerium ersichtlich sein.\\
(4) Die Gesellschaft muss monatlich dem deutschen Reichswirtschaftsministerium die Buchhaltung zukommen lassen.\\
(5) Verstöße gegen die Buchhaltungbestimmungen haben das Strafmaß gemäß \ref{apored} Abs. 3 zur Folge.

\subsection{Insolvenz} \label{apored}
(1) Verfügt eine Gesellschaft nur noch über die Hälfte des Stammkapitals, muss es Konkurs anmelden.\\
(2) Eine Gesellschaft, welche bankrott geht und zuvor nicht an einen neuen Eigentümer überschrieben wurde, verliert die Genehmigung, Geschäfte auszuüben und wird aus dem Handelsregister ausgetragen.\\
(3) Geht die Gesellschaft Absatz 1 oder 2 nicht nach, so muss der Eigentümer die Haftung ungeachtet der Rechtsform übernehmen und die Gesellschaft wird umgehend aus dem Handelsregister ausgetragen und ist somit nicht länger fähig, ihren Eigentümer zu wechseln.\\

\subsection{Internationaler Handel}
(1) Um Transaktionen in das Ausland und Inland vorzunehmen, muss man eine Zweigstelle auf deutschem Territorium im Reichshandelsregister registriert haben.\\
(2) Zweigstellen müssen über eine deutsche Rechtsform verfügen.\\
(3) Es ist deutschen Staatsbürgen untersagt, die Hauptzweigstelle im Ausland zu gründen.\\
(4) Der Buchführungspflicht gemäß \ref{buch} unterliegen jegliche deutschen Hauptzweigstellen und deren Zweigstellen, sowie jegliche Hauptzweigstellen und deren Zweigstellen, sofern sie eine Zweigstellung im Kaiserreich Deutschland haben.\\

\subsection{Haftung}
(1) Haftung für die Waren übernimmt derjenige, der sie zurzeit besitzt.\\
(2) Dies gilt sowohl auf deutschem Grunde als auch für Gesellschaften mit Zweistelle oder Hauptzweigstelle auf deutschem Grund.\\
(3) Die Haftung gegenüber dem Staat unterliegt stets dem Gesellschafter.\\
(4) Im Falle, dass der Gesellschafter nicht die juristische Person der Gesellschaft ist, trifft Absatz 3 nur dann zu, wenn die juristische Person zahlungsunfähig ist.

\subsection{Banken}
(1) Deutsche Banken müssen die Kontoinformationen der Kunden bei Anfrage durch das Reichswirtschaftsministerium offenlegen.\\
(2) Ohne Beschluss des Reichswirtschaftsministeriums dürfen sie keine Konten einfrieren.\\
(3) Deutschen Gesellschaften ist es untersagt, sich bei Banken zu registrieren, die keine Vertragspartei im Bankenabkommen oder ähnlichen Abkommen sind.\\
(4) Verstöße gegen Absatz 3 werden für Gesellschaften gemäß \ref{apored} Abs. 3 in Verbindung mit \ref{steuern} geahndet.\\

\section{Vertragsrecht}
\subsection{Grundsätzliche Regelungen}
(1) Verträge müssen der aktuellen Rechtslage des Reichs entsprechen.\\
(2) Bei Rechtswidrigkeit verliert der Vertrag seine gesamte Gültigkeit.\\
(3) Dies fällt weg sofern im Vertrag festgesetzt wurde, dass lediglich die betroffene Passage wegfällt.\\
(4) Der Vertrag muss neu aufgesetzt werden, sofern eine Passage aufgrund einer Gesetzesänderung rechtwidrig wird.
\end{document}