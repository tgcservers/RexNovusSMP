\documentclass{article}
\usepackage[utf8]{inputenc}
\usepackage{textcomp}
\usepackage{amsmath}
\usepackage{enumerate}
\usepackage{ragged2e}
\usepackage{blindtext}

\renewcommand{\thesection}{\Roman{section}}
\counterwithout{subsection}{section}
\renewcommand{\thesubsection}{§\arabic{subsection}}

\title{Discordrecht}
\author{Präsident am Obersten Servergericht}
\date{26. Juli 2023}

\begin{document}
\maketitle
\newpage
\tableofcontents
\newpage
\subsection{Chatverhalten}\label{verhalten}
(1) Spam, Beleidigungen, Drohungen und Provokationen gegen andere Spieler sind verboten und werden zu Sanktionen führen. [1] Als Spammen wird das Verschicken von mehreren Nachrichten in einem geringen Zeitintervall bezeichnet. Ab fünf Nachrichten in kürzester Zeit kann es Konsequenzen nach sich ziehen. [2] Das unerlaubte Nutzen von Pings ist aufgrund seiner provokativen Natur ebenfalls untersagt.\\
(2) [1] Rassistische, politische, ethisch inakzeptable Inhalte (Äußerungen, Bilder, etc.) sind verboten und führen zu einem permanenten Ausschluss auf dem gesamten Discord-Server. [2] Dies gilt auch für pornografische Inhalte. [3] Auch gilt dies für die absichtliche Anordnung von Reaktionen der Kategorie "Regional Indicator" zu derartigen Äußerungen. [4] Das Senden von GIFs ist verboten.\\
(3) Für pornografische Zwecke explizit angelegte Textkanäle sind von Absatz 2 Satz 2 ausgeschlossen.\\
(4) Für Memes vorgesehene Textkanäle sind von Absatz 2 Satz 4 ausgenommen.\\
(5) Weiterhin dürfen Kanäle nur für den Zweck verwendet werden, für den sie vorgesehen sind. Bestehen Unklarheiten über den Verwendungszweck, so muss man sich vor dem Verfassen einer Nachricht an den Support (\ref{support}) wenden.

\subsection{Teammitglieder}\label{members}
(1) Anweisungen von befehlsbefugten Teammitgliedern sind verbindlich und stets zu befolgen.\\
(2) Teammitglieder werden durch eine Rangbezeichnung, beziehungsweise Rolle gekennzeichnet.\\
(3) Zu den befehlsbefugten Teammitgliedern gehören:
\begin{enumerate}
	\item Der Emperor
	\item Der Oberste Serverrichter
	\item Der leitende Techniker (Lead Technician)
	\item Der Techniker
	\item Der Admin (Admin Minecraft / Admin Discord)
\end{enumerate}

\subsection{Verhalten im Sprachchat}
(1) Abgesehen von den Regelungen aus \ref{verhalten} gelten für Sprachkanäle zusätzlich folgende Bestimmungen.\\
(2) Die Nutzung von Stimmenverzerrern und Soundboards ist erlaubt, sofern die Teammitglieder keine Einwände erheben.\\
(3) Es ist nicht gestattet, Personen ohne deren Einverständnis aufzuzeichnen.

\subsection{Gültigkeit}
(1) Tritt man diesem Server bei, akzeptiert man die hier festgesetzten Bestimmungen.\\
(2) Die Server-Administration behält sich das Recht vor, diese Regeln jederzeit zu ändern.\\
(3) weggefallen\\
(4) Die Regelungen treten erst in Kraft, sobald sie in dem Textkanal für Regeln veröffentlicht werden. Dementsprechend gilt keine Regelung rückwirkend.\\
(5) Mangelnde oder fehlerhafte Kenntnisse der Serverbestimmungen gewähren keine rechtliche Immunität, da das Informieren über die aktuelle Gesetzeslage des Servers Pflicht ist.\\
(6) Ebenfalls muss man sich bei Unklarheiten an den Zuständigen wenden \ref{members}.\\
(7) Verstößt eine der Bestimmungen gegen die Verfassung des Landes einer betroffenen Person, so wird für diese lediglich die rechtswidrige Passage aufgehoben\footnote{Salvatorische Klausel}.

\subsection{Zweitaccounts}
Man muss Zweitaccount als solche markieren. Hierfür muss man sein Serverprofil derartig bearbeiten, dass es jedem möglich ist, anhand dieses Profils nachvollziehen zu können, um wessen Zweitaccount es sich handelt.

\subsection{Strafmaß}
(1) Es wird im Allgemeinen zwischen drei Strafen differenziert:
\begin{enumerate}
	\item Eine Verwarnung ist eine Vorstufe zu tatsächlichen Strafmaßnahmen. Jedes Mitglied bekommt für minder schwere Verstöße eine Verwarnung.
	\item Ein Timeout bezeichnet einen temporären Ausschluss vom Server.
	\item Ein permanenter Bann ist ein unwiderruflicher, zeitlich unbegrenzter Ausschluss vom Server.
\end{enumerate}
(2) Das Strafmaß wird selten nach der Schwere des Verstoßes, sondern zumeist nach folgender Vorgabe bemessen:
\begin{enumerate}
	\item 1. Verwarnung
	\item 2. Verwarnung
	\item 24-Stunden-Timeout
	\item Einwöchiges Timeout
	\item Ein-Monat-Timeout
	\item 1-Jahr-Timeout
	\item Permanenter Bann
\end{enumerate}
(3) Jede Strafe muss ausnahmslos widerrufen werden, sofern die bestrafte Person die Unrechtmäßigkeit der Strafe nachweisen kann.\\
(4) Unrechtmäßigkeit liegt vor, sofern es sich bei der fraglichen Tat um keinen Verstoß seitens des Bestraften handelt, bei der Bestrafung gegen Absatz 1 und 2 verstoßen wurde oder die Tat fälschlicherweise als Straftat besonderer Schwere eingestuft wurde.\\
(5) Handelt es sich bei der Tat um einen schweren Verstoß, so kann je nach Schwere des Verstoßes ein sofortiges Timeout bishin zu einem sofortigen permanenten Bann erfolgen. Die Einschätzung der Schwere unterliegt der Administration und dem Richter, muss jedoch nachvollziehbar sein. \\
(6) Sofern Zweifel bestehen, kann das Urteil von einer einfachen Mehrheit der Administration inklusive des Richters oder von den Serverownern aufgehoben und rückgängig gemacht oder in eine andere Strafe umgewandelt werden.\\
(7) Jegliche rechtswidrigen Nachrichten müssen in Form eines Screenshots bis zum Anbruch der übernächsten Woche zwischengespeichert werden, damit im Zweifelsfall die Rechtswidrigkeit angefochten werden kann\footnote{Dies begründet sich in vergangenen Schwierigkeiten, die Rechtswidrigkeit von Aussagen im Nachhinein zu bewerten.}, danach kann man bei den Serverinhabern eine Löschung beantragen, die jedoch mit einer Mehrheit bestätigt werden muss.\\
(8) Nach jedem Timeout steigt die Schwere der Straftat so, dass der erste Verstoß nach einem Timeout gemäß Strafhierarchie aus Absatz 2 Nummer 1 \- 7 aufgrund seiner Schwere im Verhältnis zur vorherigen Grundstrafe\footnote{Die erste Strafe nach einem Timeout, beziehungsweise die insgesamt erste Strafe.} erhöht wird.\\
(9) Von Absatz 8 sind permanente Banns teils ausgeschlossen. Diese sollen lediglich im Falle äußerster Schwere oder beim Zutreffen von Absatz 8 verhängt werden, sofern keine akute Verhaltensbesserung vorliegt und die Administration in einem Mehrheitsbeschluss dafür stimmt.\\
(10) Es ist nicht gestattet, entgegen der ausdrücklichen Erlaubnis der Serverbesitzer, Personen auf dem Server zu entbannen. Dies gilt als strafbar und wird ungeachtet der Position in der Strafhierarchie nach Interpretation gemäß Absatz 8 mit einem Timeout bestraft. Die widerrechtlich entbannte Person ist zudem umgehend gebannt zu werden.

\subsection{Support}\label{support}
(1) Jegliche Fragen bezüglich des Discord- und Minecraft-Servers, die nicht in den rechtlichen Bereich fallen, fallen in den Aufgabenbereich des Serversupports. Bestehen Unklarheiten bezüglich des Zuständigkeitsbereichs, sollte man sich an den Support wenden.\\
(2) Die Anschrift lautet "support@monkey-kingdom.net". Anderweitige Anschreiben sind nicht rechtskräftig und werden daher ignoriert\footnote{Zeitweise ausgesetzt}.\\
(3) Absatz 2 ist für die Zeit ausgesetzt, in der diese Email-Adresse nicht genutzt wird.

\subsection{Oberstes Servergericht}
(1) In den Aufgabenbereich des Obersten Servergerichts fallen:
\begin{enumerate}
    \item Rechtliche Fragen zur Serververfassung
    \item Anfragen rechtlichen Beistands
    \item Anfechtungen servergerichtlicher Urteile
    \item Verfassungsbeschwerden
    \item Gesetzesvorschläge
\end{enumerate}
(2) Gesetzesvorschläge, die von der Änderung oder Abschaffung bereits bestehender Gesetzer sprechen, gelten als Verfassungsbeschwerden.\\
(3) Sowohl Gesetzesverstöße und Berufung, als auch Verfassungsbeschwerden gelten als ausreichende Begründung für einen vollwertigen Prozess.\\
(4) Mitglieder des Obersten Servergerichts tragen die Amtsbezeichnung "Oberster Richter".\\
(5) Dem Obersten Gericht steht der Präsident am Obersten Servergericht vor\footnote{Gemäß Beschlusses vom 14. März 2023 ist dies der Träger der Rolle "Oberster Serverrichter". Gemäß Beschluss vom 07. Februar 2023 ist dieser Titel GodEmperor\#0001 vorbehalten.}.\\
(6) Sofern keine neuen, ausreichenden Beweise vorliegen, trifft Absatz 5 nicht zu.\\
(7) Die Anschrift lautet "ObServG@monkey-kingdom.net". Anderweitige Anschreiben sind nicht rechtskräftig und werden daher ignoriert\footnote{Zeitweise ausgesetzt}.\\
(8) Urteile müssen von dem Präsidenten am Obersten Servergericht bestätigt werden und können daher abgewiesen werden.

\subsection{Inhaber}
(1) Als Inhaber gilt, wer den Server finanziert.\\
(2) Das Stimmrecht im Inhaberkonzil entspricht den Anteilen am Server.\\
(3) Auf Anordnung der Inhaber hin, kann ein Adminkonzil einberufen werden.\\
(4) Im Adminkonzil haben sowohl die Inhaber, als auch alle Serveradministratoren eine gleichwertige Stimme.\\

\subsection{Bürgschaft}
(1) Es können nur Anmeldungen von Personen angenommen werden, für die jemand nachweislich bürgt.\\
(2) Begeht eine Person einen Verstoß besonderer Schwere, so wird sie mitsamt des Bürgenden vom Server gebannt.\\
(3) Wird ein Bürgender vom Server gebannt, geschieht dies denen gleich, für die dieser bürgt.\\
(4) Bürgschaften kann man nicht nachträglich zurückziehen.\\
(5) Die Administration ist von Abs. 2f. ausgenommen.\\

\subsection{Rechtliche Separation}
(1) Das Serverrecht ist eindeutig von der internen Rechtssituation auf den Ablegern des Monkey Kingdom Servers zu unterscheiden.\\
(2) Als internes Recht werden nicht von der Inhaberschaft in ihrer Funktion als Inhaberkonzil anerkannte Verfassungen und Regeln, wie beispielsweise fraktionseigene Gesetzestexte bezeichnet.\\
(3) Die Einsicht und Nutzung von, internen Regelungen übergeordneten, Serverdaten und sonstigen, nur für die Administration zugänglichen Informationen, wie Spielerdaten oder Logs, darf nicht zur Beweisführung für Prozesse und ähnliches verwendet werden, die nicht von dem Obersten Servergericht in dessen Funktion vollzogen werden\footnote{So dürfen beispielsweise In-Game-Morde nicht über Logs nachgewiesen werden}.
\end{document}