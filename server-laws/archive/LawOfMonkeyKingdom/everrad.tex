\documentclass{article}
\usepackage[utf8]{inputenc}
\usepackage{enumerate}
\usepackage{ragged2e}

\renewcommand{\thesection}{\Roman{section}}
\counterwithout{subsection}{section}
\renewcommand{\thesubsection}{§\arabic{subsection}}

\title{Codex Radanas}
\author{Radanas Rayaradan Irismar}
\date{19. Juni 1991}

\begin{document}
\maketitle
\vspace*{\fill}
\paragraph{Verfassung von Everrad}

\newpage
\topskip0pt
\vspace*{\fill}
\begin{Center}
\textbf{1. Fassung}
\end{Center}
\begin{flushright}
\textit{Berlin, den 19. Juni 2023}
\end{flushright}
\textbf{\ldots}
\\\\
\textbf{Radanas Rayaradan Irismar}
\\\\\\\\\\\\\\\\\\\\\\\\\\\\\\\\
\vspace*{\fill}

\newpage
\tableofcontents
\newpage
\section{Tabula prima: De legum}
\subsection{Lex votum motivum}
Die \textit{lex votum motivum} besagt, dass ein Amt in einer Versammlung bei Stimmgleichheit eine zweite Stimme erhält.

\subsection{Lex votum maiestatis}
Einem führenden Amt wird die Fähigkeit zugesichert, ohne Begründung ein Majestätsvotum zu veranlassen, bei welchem nur ein geringer Kreis an Mitgliedern teilnehmen darf.

\subsection{Lex votum imperatoris}
Dem Fürsttribun steht es zu, Beschlüsse niederrangiger Instanzen zu annullieren. Dieses Recht steht dem Hochkonzil nicht zu.

\subsection{Annuitätsprinzip}
Das Annuitätsprinzip besagt, dass eine Amtszeit stets ein Jahr dauern muss.

\subsection{Sitte des Prinzipats}
Es steht dem Fürsttribun nicht zu, sich als erhabener oder majestätischer als der Hochkonzil darzustellen.

\subsection{Mos consensus}
Dem Fürsttribun ist es nicht gestattet, sich über Beschlüsse des Hochkonzils hinwegzusetzen oder sich zum ewigen Anführer auszurufen.

\subsection{Passivus est activus}
Jegliche verbotene aktive Tat ist auch als passive Tat untersagt.

\subsection{Clausula praesidii legis}
Mangelnde oder fehlerhafte Kenntnisse der Rechtslage gewähren keine rechtliche Immunität.

\subsection{Clausula absentiae}
Bei selbstverschuldeter und unentschuldbarer Abwesenheit vor Gericht, dürfen Prozesse in Abwesenheit der fehlenden Partei abgehalten werden.

\subsection{Clausula reverentiae}
Man muss der Richterschaft Respekt zollen.

\section{Tabula secunda: De re publica}
\subsection{Der Hochkonzil}
Die oberste Gewalt des Stadtstaats ist der Hochkonzil. Diesem kommen die Befugnisse der höchsten ordentlichen Gerichtsbarkeit und der Verfassungsgebung zu. Die Mitglieder nennen sich Tribune. In seiner Form als Gericht wird es als \textit{tribunal dignitatis} bezeichnet.

\subsection{Der Fürsttribun}
Das Haupt des Hochkonzils ist der Fürsttribun. Dieser verfügt über die \textit{lex votum motivum}.

\section{Tabula tertia: De delicto}
\end{document}