\documentclass{article}
\usepackage[utf8]{inputenc}
\usepackage{textcomp}
\usepackage{amsmath}
\usepackage{enumerate}
\usepackage{ragged2e}
\usepackage{blindtext}

\renewcommand{\thesection}{\Roman{section}}
\counterwithout{subsection}{section}
\renewcommand{\thesubsection}{§\arabic{subsection}}

\title{Richtlinien von Reign SMP}
\author{the.god.emperor}
\date{22. Dezember 2023}

\begin{document}
\maketitle
\newpage
\tableofcontents
\newpage
\section{Discord-Server}
\subsection{Chatverhalten}\label{verhalten}
\begin{enumerate}[(1)]
	\item $^{1}$Spam, Beleidigungen, Drohungen und Provokationen gegen andere Spieler sind verboten und werden zu Sanktionen führen. $^{2}$Als Spammen wird das Verschicken von mehreren Nachrichten in einem geringen Zeitintervall bezeichnet. Ab fünf Nachrichten in kürzester Zeit kann es Konsequenzen nach sich ziehen. $^{3}$Das unerlaubte Nutzen von Pings ist aufgrund seiner provokanten Natur ebenfalls untersagt.
	\item $^{1}$Rassistische, politische, ethisch inakzeptable Inhalte (Äußerungen, Bilder, etc.) sind verboten und führen zu einem permanenten Ausschluss auf dem gesamten Discord-Server. $^{2}$Dies gilt auch für pornografische Inhalte. $^{3}$Auch gilt dies für die absichtliche Anordnung von Reaktionen der Kategorie "Regional Indicator" zu derartigen Äußerungen. $^{4}$Das Senden von GIFs ist verboten.
	\item Für pornografische Zwecke explizit angelegte Textkanäle sind von Absatz 2 Satz 2 ausgeschlossen.
	\item Für Memes vorgesehene Textkanäle sind von Absatz 2 Satz 4 ausgenommen.
	\item Weiterhin dürfen Kanäle nur für den Zweck verwendet werden, für den sie vorgesehen sind. Bestehen Unklarheiten über den Verwendungszweck, so muss man sich vor dem Verfassen einer Nachricht an den Support (\ref{support}) wenden.
\end{enumerate}

\subsection{Teammitglieder}\label{members}
\begin{enumerate}[(1)]
	\item Anweisungen von befehlsbefugten Teammitgliedern sind verbindlich und stets zu befolgen.
	\item Teammitglieder werden durch eine Rangbezeichnung, beziehungsweise Rolle gekennzeichnet.
	\item Zu den befehlsbefugten Teammitgliedern gehören:
	\begin{enumerate}
		\item Der CEO
		\item Der CLO
		\item Der CAO
		\item Der COO
		\item Der CTO
		\item Die Shareholder
		\item Die DC-Administratoren
		\item Die Techniker
		\item Die DC-Moderatoren
		\item Die MC-Administratoren
		\item Die MC-Moderatoren
	\end{enumerate}
	\item Die befehlsbefugten Mitglieder dürfen nur Befehle in ihrem Zuständigkeitsbereich erteilen.
	\item Der Missbrauch von Befugnissen ist regelwidrig.
\end{enumerate}

\subsection{Verhalten im Sprachchat}
\begin{enumerate}[(1)]
	\item Abgesehen von den Regelungen aus \ref{verhalten} gelten für Sprachkanäle zusätzlich folgende Bestimmungen.
	\item Die Nutzung von Stimmenverzerrern und Soundboards ist erlaubt, sofern die Teammitglieder keine Einwände erheben.
	\item Es ist nicht gestattet, Personen ohne deren Einverständnis aufzuzeichnen.
	\item Fehlverhalten in Sprachkanälen rechtfertigt eine temporäre serverweite Stummschaltung.
\end{enumerate}

\subsection{Gültigkeit}
\begin{enumerate}[(1)]
	\item Tritt man diesem Server bei, akzeptiert man die hier festgesetzten Bestimmungen.
	\item Die Server-Administration behält sich das Recht vor, diese Regeln jederzeit zu ändern.
	\item \textit{weggefallen}
	\item Die Regelungen treten erst in Kraft, sobald sie in dem Textkanal für Regeln veröffentlicht werden. Dementsprechend gilt keine Regelung rückwirkend.
	\item Mangelnde oder fehlerhafte Kenntnisse der Serverbestimmungen gewähren keine rechtliche Immunität, da das Informieren über die aktuelle Gesetzeslage des Servers Pflicht ist.
	\item Ebenfalls muss man sich bei Unklarheiten an den Zuständigen wenden (\ref{members}).
	\item Verstößt eine der Bestimmungen gegen die Verfassung des Landes einer betroffenen Person, so wird für diese lediglich die rechtswidrige Passage aufgehoben\footnote{Salvatorische Klausel}.	
\end{enumerate}

\subsection{Zweitaccounts}
Man muss Zweitaccounts als solche markieren. Hierfür muss man sein Serverprofil derartig bearbeiten, dass es jedem möglich ist, anhand dieses Profils nachvollziehen zu können, um wessen Zweitaccount es sich handelt.

\subsection{Strafmaß}
\begin{enumerate}[(1)]
	\item Es wird im Allgemeinen zwischen drei Strafen differenziert:
	\begin{enumerate}[1.]
		\item Eine Verwarnung ist eine Vorstufe zu tatsächlichen Strafmaßnahmen. Jedes Mitglied bekommt für minder schwere Verstöße eine Verwarnung.
		\item Ein Timeout bezeichnet einen temporären Ausschluss vom Server.
		\item Ein permanenter Bann ist ein unwiderruflicher, zeitlich unbegrenzter Ausschluss vom Server.
	\end{enumerate}
	\item Das Strafmaß wird selten nach der Schwere des Verstoßes, sondern zumeist nach folgender Vorgabe bemessen:
	\begin{enumerate}[1.]
		\item Erste Verwarnung
		\item Zweite Verwarnung
		\item 24-Stunden-Timeout
		\item Einwöchiges Timeout
		\item Ein-Monat-Timeout
		\item 1-Jahr-Timeout
		\item Permanenter Bann
	\end{enumerate}
	\item Jede Strafe muss ausnahmslos widerrufen werden, sofern die bestrafte Person die Unrechtmäßigkeit der Strafe nachweisen kann.
	\item Unrechtmäßigkeit liegt vor, sofern es sich bei der fraglichen Tat um keinen Verstoß seitens des Bestraften handelt, bei der Bestrafung gegen Absatz 1 und 2 verstoßen wurde oder die Tat fälschlicherweise als Straftat besonderer Schwere eingestuft wurde.
	\item Handelt es sich bei der Tat um einen schweren Verstoß, so kann je nach Schwere des Verstoßes ein sofortiges Timeout bishin zu einem sofortigen permanenten Bann erfolgen. Die Einschätzung der Schwere unterliegt dem Zuständigen, muss jedoch nachvollziehbar sein. 
	\item Sofern Zweifel bestehen, kann das Urteil von dem CLO oder durch eine qualifizierte Mehrheit durch den Vorstand aufgehoben und rückgängig gemacht oder in eine andere Strafe umgewandelt werden.
	\item Jegliche rechtswidrigen Nachrichten müssen in Form eines Screenshots bis zum Anbruch der übernächsten Woche zwischengespeichert werden, damit im Zweifelsfall die Rechtswidrigkeit angefochten werden kann\footnote{Dies begründet sich in vergangenen Schwierigkeiten, die Rechtswidrigkeit von Aussagen im Nachhinein zu bewerten.}, danach kann man bei dem Vorstand eine Löschung beantragen, die jedoch mit einer qualifizierten Mehrheit bestätigt werden muss.
	\item Nach jedem Timeout steigt die Schwere der Straftat so, dass der erste Verstoß nach einem Timeout gemäß Strafhierarchie aus Absatz 2 Nummer 1 - 7 aufgrund seiner Schwere im Verhältnis zur vorherigen Grundstrafe\footnote{Die erste Strafe nach einem Timeout, beziehungsweise die insgesamt erste Strafe.} erhöht wird.
	\item Von Absatz 8 sind permanente Banns teils ausgeschlossen. Diese sollen lediglich im Falle äußerster Schwere oder beim Zutreffen von Absatz 8 verhängt werden, sofern keine akute Verhaltensbesserung vorliegt und der Vorstand in einem einfachen Mehrheitsbeschluss dafür stimmt.
	\item Es ist nicht gestattet, entgegen der ausdrücklichen Erlaubnis des CLO, Personen auf dem Server zu entbannen. Dies gilt als strafbar und wird ungeachtet der Position in der Strafhierarchie nach Interpretation gemäß Absatz 8 mit einem Timeout bestraft. Die widerrechtlich entbannte Person ist zudem umgehend gebannt zu werden.
	\item $^{1}$Verstöße besonderer Schwere durch Mitglieder eines Unternehmensbereichs resultieren in einem permantenten Ausschluss aus diesem und weiteren Unternehmensbereichen. $^{2}$Von dieser Bestimmung sind the.god.emperor und seine Zweitaccounts ausgeschlossen.
	\item Mittäterschaft, Beihilfe, Anstiftung und Versuch werden äquivalent betraft.
\end{enumerate}

\subsection{Inhaber}
\begin{enumerate}[(1)]
	\item Der Begriff des Inhabers, beziehungsweise Shareholders, entspricht dem Begriff des Teilhabers gemäß §3 Abs. 1 TeilhB.
	\item Jegliche Beschlüsse der Inhaberschaft erfordern eine qualifizierte Mehrheit, es sei denn, die Teilhaberschaftsbestimmungen widersprechen dem.
	\item Auf Anordnung der Inhaber hin kann ein Adminkonzil einberufen werden, bei welchem die Admins und Moderatoren, die keine Teilhaber sind, je eine Stimme bekommen und die Entscheidung für die Teilhaberversammlung übernehmen.
\end{enumerate}

\subsection{Vorstandsvorsitz}
\begin{enumerate}[(1)]
	\item Den Vorstandsvorsitz gemäß §8 Abs. 2 TeilhB nimmt der Chief executive officer ein. Diese Rangbezeichnung wird mit `CEO' abgekürzt.
	\item Er wird durch die Aktionärsversammlung mittels einer einfachen Mehrheit gewählt.
	\item \textit{weggefallen}
	\item Der Chief executive officer kann nur aus den Reihen der Aktionäre gewählt werden.
	\item Der Vorstandsvorsitz ist Transaktionsberechtigter gemäß §7 TeilhB.
	\item Der CEO wählt den Vorsitz der Unternehmensbereiche. Hierbei kann er nur Personen wählen, die bereits in der Abteilung arbeiten.
\end{enumerate}

\subsection{Unternehmensbereiche}
\begin{enumerate}[(1)]
	\item Die nachfolgenden Unternehmensbereiche gemäß §9 TeilhB bestehen auf dem Server:
		\begin{enumerate}[1.]
            \item Administration
			\item Rechtsabteilung (Legal Department)
			\item Moderation
			\item Technische Abteilung (Technical Department)
		\end{enumerate}
	\item An den Vorsitz dieser Abteilungen wird die dementsprechende Rolle vergeben.
	\item In den Unternehmensbereichen dürfen nur Personen angestellt werden, die auf dem jeweiligen Gebiet über ausreichende Kenntnisse und Fähigkeiten verfügen.
\end{enumerate}

\subsection{Rechtsabteilung}
\begin{enumerate}[(1)]
	\item In den Aufgabenbereich der Rechtsabteilung fallen:
	\begin{enumerate}[1.]
		\item Rechtliche Fragen zur Serververfassung
		\item Anfragen rechtlichen Beistands
		\item Anfechtungen servergerichtlicher und sonstiger Urteile
		\item Rechtliche Beschwerden
		\item Gesetzesvorschläge
		\item Behandlung von Verstößen gegen Serverrichtlinien
		\item Behandlung von Verstößen gegen die Teilhaberschaftsbestimmungen
		\item Prüfung der Urteile der Moderation
	\end{enumerate}
	\item Gesetzesvorschläge, die von der Änderung oder Abschaffung bereits bestehender Bestimmungen sprechen, gelten als rechtliche Beschwerden.
	\item Sowohl Regelverstöße und Berufung, als auch rechtliche BEschwerden gelten als ausreichende Begründung für eine vollwertige Anhörung.
	\item Der Vorsitzende der Abteilung ist der Chief legal officer (CLO).
	\item Mitglieder der Rechtsabteilung werden als `Server lawyer' (Serveranwalt) bezeichnet.
	\item Sofern keine neuen, ausreichenden Beweise vorliegen, trifft Absatz 3 nicht zu.
	\item Urteile durch die Moderation und die Rechtsabteilung müssen von dem CLO bestätigt werden und können daher abgewiesen werden.
	\item Die Abweisung von Urteilen muss gerechtfertigt sein und begründet werden.
\end{enumerate}

\subsection{Moderation}\label{support}
\begin{enumerate}[(1)]
	\item Jegliche Fragen bezüglich des Discord- und Minecraft-Servers, die nicht in den rechtlichen Bereich fallen, fallen in den Aufgabenbereich der Moderation. Bestehen Unklarheiten bezüglich des Zuständigkeitsbereichs, sollte man sich ebenfalls an die Moderation wenden.
	\item Die Moderationsabteilung dient zur Kontrolle der Einhaltung der Serverrichtlinien.
	\item Dies bedingt, dass sie in der Lage sind, ohne eine Genehmigung Strafen zu vollziehen, die allerdings an die Rechtsabteilung mitsamt des Kontexts weitergeleitet werden und endgültig bestätigt werden müssen.
	\item Den Vorstand der Moderation hat der Chief operating officer (COO).
	\item Von Absatz 3 ausgeschlossen sind jegliche Verstöße gegen die Teilhaberschaftsbestimmungen, da diese von der Rechtsabteilung und VIRTSTAX behandelt werden.
	\item Mitglieder der Moderation werden je nach Zuständigkeitsbereich als `MC Moderator' (Minecraft-Moderator) oder als `DC Moderator' (Discord-Moderator) aufgeführt.
	\item Moderatoren dürfen Urteile gemäß Absatz 3 nur innerhalb ihres Zuständigkeitsbereichs vollziehen.
	\item Moderatoren verfügen über keine administrativen Berechtigungen.
\end{enumerate}

\subsubsection{Administration}
\begin{enumerate}[(1)]
    \item Die Administration dient der Verwaltung und Koordinierung des Servers und der Belegschaft.
    \item Ihm sitzt der Chief administrative officer (CAO) vor.
    \item Administrative Mitarbeiter auf dem Discord Server sind als `DC Admin' (Discord-Administrator) gekennzeichnet und verfügen über administrative Berechtigungen auf dem Discord-Server.
    \item Mitglieder der Administration, die für den Minecraft-Server zuständig sind, werden als `MC Admin' (Minecraft-Admin) bezeichnet.
    \item Administrative Mitglieder müssen durch eine qualifizierte Mehrheit des Vorstands bestätigt werden.
\end{enumerate}

\subsection{Technische Abteilung}
\begin{enumerate}[(1)]
	\item Die technische Abteilung dient der Wartung aller Server und Dienstleistungen, die dem Reign SMP unterstehen.
	\item Dies bezieht auch die Realisierung neuer Funktionalitäten auf diesen Servern ein.
	\item Den Vorsitz hat der Chief technology officer (CTO).
	\item Mitglieder dieser Abteilung werden als `Technicians' (Techniker) bezeichnet.
\end{enumerate}

\section{Minecraft-Server}

\subsection{Bürgschaft}
\begin{enumerate}[(1)]
	\item Es dürfen nur Personen auf die Whitelist gesetzt werden, für ein Mitglied nachweislich bürgt.
	\item Begeht eine Person einen Verstoß besonderer Schwere, so wird sie mitsamt des Bürgenden vom Server gebannt.
	\item Wird ein Bürgender vom Server gebannt, geschieht dies denen gleich, für die dieser bürgt.
	\item Bürgschaften kann man nicht nachträglich zurückziehen.
	\item Der Vorstand ist von Abs. 2f. ausgenommen.
\end{enumerate}

\subsection{Rechtliche Separation}
\begin{enumerate}[(1)]
	\item Das Serverrecht ist eindeutig von der internen Rechtssituation auf den Ablegern des Reign SMP Servers zu unterscheiden.
	\item Als internes Recht werden nicht von der Inhaberschaft in ihrer Funktion als Teilhaberversammlung anerkannte Verfassungen und Regeln, wie beispielsweise fraktionseigene Gesetzestexte bezeichnet.
	\item Die Einsicht und Nutzung von, internen Regelungen übergeordneten, Serverdaten und sonstigen, nur für die Administration zugänglichen Informationen, wie Spielerdaten oder Logs, darf nicht zur Beweisführung für Prozesse und ähnliches dienen, die nicht von der Rechtsabteilung in ihrer Funktion vollzogen werden\footnote{So dürfen beispielsweise In-Game-Morde nicht über Logs nachgewiesen werden}.
\end{enumerate}

\subsection{Grundsätzliche Regeln}
\begin{enumerate}[(1)]
	\item Das Minecraft-Serverrecht untersteht dem Discordserverrecht.
	\item Es ist verboten, auf Methoden zurückzugreifen, die gegenüber anderen Spielern, ungeachtet dessen, ob sie die Methode einsetzen oder nicht, einen Vorteil verschaffen, die allgemein nicht als gerecht anerkannt werden.
	\item Jegliches, von derartigen Methoden nicht betroffenes Verhalten, ist nicht strafbar.
	\item Von Absatz 3 ist der Bau von Konstruktionen, die dem Zwecke dienen, die Serverleistung zu verringern, ausgeschlossen.
	\item Auf dem Minecraft-Server muss man sich den jeweiligen Regeln des Discord-Servers entsprechend verhalten.
	\item Das generelle Serverrecht unterscheidet nicht zwischen Fraktionen, weshalb diese lediglich eine interne Organisation darstellt, die keine Deckung durch jegliche serverweite Gesetze erfährt und somit Verbrechen gegen diese im Einzelnen kein Gegenstand serverweiter Urteile sein können.
	\item Absatz 3 tritt nur ein, wenn sich die Verstöße nicht gegen die Regeln des Discordservers oder Abs. 1f. richten.
\end{enumerate}

\subsection{Fraktionen}
\begin{enumerate}[(1)]
	\item Als Fraktion gilt jegliche Gruppierung mit mehr als einem Spieler.
	\item Der Begriff des Spielers ist nicht mit dem Begriff des Minecraft-Kontos synonym und rechtfertigt daher keine Gründung, wenn es sich bei dem anderen Konto um ein Zweitkonto der Person handelt.
	\item Fraktionen haben das Anrecht auf eine eigene Kategorie, in der sie jegliche Kanäle auf Anfrage hin einrichten können.
	\item Die Moderation kann derartige Anliegen ablehnen, sofern diese keinen gerechtfertigten Grund für eine Einrichtung feststellen können.
	\item Aufgrund der Tätigkeit und Aufgaben der Moderation ist diese jederzeit berechtigt, in die Kanäle einzusehen, um Verstöße gegen das geltende Recht erkennen zu können.
	\item Jeder Fraktion wird eine eigene Rolle zugesichert, die zum Zweck hat, dass diese Kanäle nicht durch Mitglieder anderer Fraktionen eingesehen werden können.
\end{enumerate}

\end{document}