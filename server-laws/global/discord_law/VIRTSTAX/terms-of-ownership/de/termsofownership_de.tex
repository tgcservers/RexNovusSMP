\documentclass{article}
\usepackage[utf8]{inputenc}
\usepackage{textcomp}
\usepackage{amsmath}
\usepackage{enumerate}
\usepackage{ragged2e}
\usepackage{blindtext}

\renewcommand{\thesection}{\Roman{section}}
\counterwithout{subsection}{section}
\renewcommand{\thesubsection}{§\arabic{subsection}}

\title{Teilhaberschaftsbestimmungen (TeilhB)}
\author{VIRTSTAX}
\date{14. August 2023}

\begin{document}
\maketitle
\newpage
\tableofcontents
\newpage
\subsection{Grundbestimmungen}
\begin{enumerate}[(1)]
	\item Die Nutzung der Teilhaberschaftsbestimmungen erfordert die Übereinstimmung der Serverstruktur mit dem aktuellsten Stand der VIRTSTAX-Spezifikationen, sowie eine offizielle Partnerschaft mit VIRTSTAX.\@
	\item Damit die Teilhaberschaftsbestimmungen ihre rechtmäßige Wirkung erlangen, muss ihre Einführung eindeutig und schriftlich von dem Serverinhaber laut Discord genehmigt worden sein.
	\item Zur Genehmigung reicht aus, dass der Serverinhaber diese Grundbestimmungen auf dem betroffenen Discord hochgeladen hat.
\end{enumerate}

\subsection{Rechte des Eigentümers laut Discord oder anderen Plattformen}
\begin{enumerate}[(1)]
	\item Sobald der Eigentümer Anteile an seinem Discordserver oder anderen Dienstleistungen seiner Organisation verkauft, verliert er die Anrechte an den verkauften Anteilen vollständig und verfügt somit nur noch über eine Teilhaberschaft.
	\item Dies bezieht ein, dass der Servereigentümer das Anrecht verliert, sich auf seinen Status als ursprünglicher Gründer oder durch Übertragung der Inhaberschaft mittels der von Discord oder anderen Plattformen zur Verfügung gestellten Methode, zu berufen und somit ein alleiniges oder den Anteilen nicht entsprechendes Stimmrecht zu rechtfertigen.
	\item Eine Serverlöschung muss ebenfalls durch die Mehrheit aller Anteilseigner genehmigt werden.
\end{enumerate}

\subsection{Allgemeine Eigentumsrechte}
\begin{enumerate}[(1)]
	\item Alle Personen, die VIRTSTAX-zertifizierte Anteile an der Organisation erworben haben, sind für den Zeitraum, in dem die Anteile gültig sind, vollwertige Teilhaber.
	\item Hat man diese Anteile nicht gemäß Absatz 1 erworben, hat man keine Anteilsrechte an der Organisation.
	\item $^{1}$Das Stimmrecht der Teilhaber entspricht ihren Anteilen an der Organisation. $^{2}$Dies bedingt ebenfalls, dass sie ihren Anteilen entsprechende Haftung für die Organisation übernehmen und im Falle rechtlicher Verfahren ihren Anteilen entsprechend zur Rechenschaft gezogen werden müssen.
	\item $^{1}$Anteilhaber mit mindestens 25.1\% gelten als Hauptverantwortliche für die Organisation und können Beschlüsse bei einer verlangten, qualifizierten Minderheit, verhindern. $^{2}$In rechtlichen Fragen sind sie die Vorsitzenden der Organisation und haben sich dementsprechend bei rechtlichen Problemen vor Gericht oder anderen Ausschüssen zu verantworten.
	\item Als vollwertiger Inhaber der Anteile kann man die Anteile ohne Genehmigung durch andere Teilhaber weiterverkaufen, muss VIRTSTAX jedoch bezüglich des Verkaufs informieren und den Verkauf autorisieren lassen.
	\item VIRTSTAX kann den Verkauf nur dann ablehnen, wenn die Aktien den ursprünglich erworbenen Aktien nicht der Richtigkeit, Gültigkeit oder Echtheit, wie sie durch VIRTSTAX vorgeschrieben werden, entsprechen, der Empfänger der Anteilshaberschaft auf der Liste verbotener Empfänger der Organisation oder von VIRTSTAX eingetragen wurde, oder der Preis geringer als der Aktiengrundwert ist.
	\item Ebenfalls entscheidet der Anteilseigner über die Kosten seiner Anteile, jedoch kann die Versammlung der Inhaber mit einer einfachen Mehrheit entscheiden, dass diese Aktien den durchschnittlichen Aktienpreisen des Monats entsprechen müssen. Mitglieder können bei dieser Entscheidung nicht von den Rechten aus Absatz 3 Satz 1 Gebrauch machen. Ebenfalls ist besagter Anteilseigner bei dieser Entscheidung nicht stimmberechtigt.
	\item Der Verkäufer ist verpflichtet, den Aktiengrundwert an die Organisation zu zahlen, dessen Aktien er verkauft hat.
	\item Anteilseigner haben eine Freigabe für alle Dokumente und sonstigen Informationen der Organisation, dürfen diese jedoch nicht ohne Genehmigung der zuständigen Instanz an Personen weitergeben, die über keine Freigabe für die fraglichen Informationen verfügen.
	\item Das in Absatz 8 beschriebene Anrecht auf Informationen gilt nicht für strukturkritische Informationen, wie den Zugang zu dem Root-Benutzer eines Servers. Derartige Informationen dürfen nur an technisches Personal und den Vorstandsvorsitz verliehen werden.
	\item Absatz 8 kann verwehrt werden, sofern ein gerechtfertigtes, glaubhaftes und nachvollziehbares Interesse an dem Rückhalt der Informationen besteht.
	\item Eigentümer dürfen nicht gezwungen werden, ihre Anteile zu verkaufen oder anderweitig aus dem Vorstand ausgeschlossen werden, sofern keine Verstöße gegen die geltenden Richtlinien der Organisation durch den Anteilseigner bestehen, die ein solches Handeln rechtfertigen würden.
\end{enumerate}

\subsection{Aktiengrundwert}
Der Aktiengrundwert bezeichnet den Mindestwert einer Aktie. Diese sind ein Tausendstel der Gesamtkosten, die für eine Organisation monatlich aufkommen.

\subsection{Aktie}
\begin{enumerate}[(1)]
	\item Als Aktie wird in den Teilhaberschaftsbestimmungen ein, von VIRTSTAX zertifiziertes Dokument bezeichnet, das die Teilhaberschaft und dessen Bestimmungen beurkundet.
	\item Das Dokument muss nachfolgende Informationen beinhalten:
	\begin{enumerate}[1.]
		\item Datum des Kaufs und der Fälligkeit
		\item Schriftliche Zertifikation durch VIRTSTAX
		\item Den Namen der übergeordneten Organisation, dessen Anteile man erworben hat
		\item Die Registrationsnummer der Aktie: Sie dient als Referenz, die wievielte Aktie dieser Organisation es ist, die von VIRTSTAX zertifiziert wurde und beinhält ebenfalls die VIRTSTAX-Abkürzung der Organisation
		\item Die Anteilswerte: Der Wert einer einzelnen Aktie in einer, durch den ISO-4217-Code angegebenen, Währung
		\item Die Anzahl der erworbenen Aktien
		\item Eine Kurzbeschreibung der Anteilsberechtigungen
		\item Signatur der Transaktionsberechtigten mitsamt derer Funktionen in der Organisation: Als Signatur darf zur Wahrung der Privatsphäre nur eine symbolische und nicht die tatsächliche Unterschrift benutzt werden
		\item Das Token: Das Token ist die eindeutige Sicherheitsnummer, die die Echtheit zertifiziert
	\end{enumerate}
	\item Nur VIRTSTAX darf Aktien ausstellen.
	\item Der Kaufpreis wird gemäß DIN 1333 kaufmännisch gerundet.
	\item Aktien dürfen nur auf offiziellen VIRTSTAX-Handelsplätzen erworben und verkauft werden.
	\item Handelsplätze von Organisationen, die eine offizielle Partnerschaft mit VIRTSTAX haben, sind keine offiziellen Handelsplätze, sofern diese nicht als solche eingetragen sind.
\end{enumerate}

\subsection{Anteilsfälligkeit}
\begin{enumerate}[(1)]
	\item Jede Aktie ist nur in dem Zeitraum gültig, für den man die Anteile bezahlt hat.
	\item Dieser Zeitraum wird meist in Monaten gemessen.
	\item Man kann Aktien für den entsprechenden Betrag bereits für längere Zeiträume erwerben.
	\item Verfällt die Gültigkeit einer Aktie, verfallen auch die Anteile, weshalb diese bei bestehendem Interesse an der Wahrung der Anteilsrechte rechtzeitig erneut erworben werden müssen.
	\item Der Zeitraum der Gültigkeit beginnt an dem Tag, an dem die Aktie ausgestellt wurde.
\end{enumerate}

\subsection{Organisation}
Als Organisation wird in den Teilhaberschaftsbestimmungen die übergeordnete Instanz bezeichnet, an der man durch den Erwerb von Aktien, Anteile erwirbt.

\subsection{Transaktionsberechtigter}
\begin{enumerate}[(1)]
	\item Ein Transaktionsberechtigter (auf der Aktie als stock transaction trustee bezeichnet) ist der, von der Inhaberschaft bevollmächtigte Treuhänder der Aktien und somit bevollmächtigt, die Ausstellung freier Aktienanteile vorzunehmen.
	\item Der Transaktionsberechtigte ist nicht in der Lage, nachfolgende Aktientransaktionen zu unterbinden, verwaltet diese allerdings dennoch.
	\item Er dient als Vermittler zwischen VIRTSTAX und dem Verkäufer und muss bestätigen, dass die Zahlung eingegangen ist, damit VIRTSTAX die Aktie ausstellen kann.
	\item In seiner Abwesenheit muss von dem Vorstand ein stellvertretender Transaktionsberechtigter gewählt werden.
\end{enumerate}

\subsection{Vorstand}
\begin{enumerate}[(1)]
	\item Der Vorstand ist die Versammlung aller Anteilseigner gemäß Teilhaberschaftsbestimmungen.
	\item Jegliche Entscheidungen, die die Organisation oder untergeordnete Instanzen betreffen und nicht ausdrücklich in einzelne Unternehmensbereiche fallen, muss durch eine Mehrheit im Vorstand genehmigt werden.
	\item Der Organisation steht es frei, die Belegung des Vorstandsvorsitzes zu regeln.
\end{enumerate}

\subsection{Unternehmensbereich}
\begin{enumerate}[(1)]
	\item Als Unternehmensbereich wird in Organisationen eine Abteilung bezeichnet, die für die Ausführung bestimmter, durch das Serverrecht ihr zugeteilter Aufgaben zuständig ist.
	\item Entscheidungen, die in den Aufgabenbereich einer Abteilung fallen und Instanzen betrifft, die dem Abteilungsleiter oder der Abteilung selbst untergeordnet sind, können ohne Genehmigung durch den Vorstand getroffen werden.
	\item Der Vorstand ist allerdings in der Lage Entscheidungen, die gemäß Absatz 2 getroffen wurden, mit einer qualifizierten Mehrheit aufzuheben.
\end{enumerate}

\end{document}