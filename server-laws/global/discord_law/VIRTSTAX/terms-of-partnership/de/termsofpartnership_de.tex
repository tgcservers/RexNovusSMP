\documentclass{article}
\usepackage[utf8]{inputenc}
\usepackage{textcomp}
\usepackage{amsmath}
\usepackage{enumerate}
\usepackage{ragged2e}
\usepackage{blindtext}

\renewcommand{\thesection}{\Roman{section}}
\counterwithout{subsection}{section}
\renewcommand{\thesubsection}{§\arabic{subsection}}

\title{Partnerschaftsbestimmungen}
\author{VIRTSTAX}
\date{14. August 2023}

\begin{document}
\maketitle
\newpage
\tableofcontents
\newpage
\subsection{Grundbestimmungen}\label{grundb}
\begin{enumerate}[(1)]
	\item Die Partnerschaftsbestimmungen gelten für alle offiziellen Partner von VIRTSTAX und alle, die sich für eine offizielle Partnerschaft bewerben.
	\item Verstöße gegen die Partnerschaftsbestimmungen werden je nach Schwere mit einer Verwarnung, einer temporären, bis hin zu einer permanenten Disqualifikation als Partner geahndet.
	\item Bei nachgewiesener Unrechtmäßigkeit der Bestrafung muss VIRTSTAX diese Bestrafung umgehend rückgängig machen und dies offiziell ankündigen.
	\item Um ein Partner zu werden, muss man den Großteil seiner Kommunikation über den eigenen Discord-Server laufen lassen. 
	\item Akzeptiert VIRTSTAX eine Bewerbung, so ist die Partnerschaft vollständig gültig und erfordert keine weitere Zustimmung durch den Bewerber.
	\item Man kann die Partnerschaft mit VIRTSTAX jederzeit aufkündigen.
	\item Beschließt VIRTSTAX, dass die Partnerschaft nur in einem periodischen Intervall aufgekündigt werden darf, so werden alle Partner, die diesen Status vor Einführung der neuen Kündigungsregelungen erlangt haben, diesen Regeln unterliegen und erst in der Lage sein, am offiziellen Beginn dieses Intervalls diese aufzukündigen.
	\item Organisationen haften für jegliche finanziellen und sonstigen Schäden, die durch den Kauf derer Aktien aufkommen.
	\item Wer die Haftung innerhalb der Organisation übernimmt, unterliegt den Regeln der Organisation.
	\item Bestehen keine Regeln diesbezüglich, so übernehmen es die Anteilseigner der Organisation.
	\item Die Haftung übernimmt die Organisation nicht, sofern VIRTSTAX die Schäden trotz besseren Wissens verursacht hat.
	\item Es dürfen nur von der VIRTSTAX offziell anerkannte Handelsplätze Aktien verkaufen.
	\item Ein offizieller Partner darf nur dann einen solchen Handelsplatz einrichten, wenn VIRTSTAX dies offiziell genehmigt und dies auch nur auf dem Discord-Server, der für diesen Partner eingetragen wurde.
\end{enumerate}

\subsection{Bewerbung}
\begin{enumerate}[(1)]
    \item Bewerben kann man sich über ein Formular, welches von VIRTSTAX zur Verfügung gestellt wird.
    \item Die VIRTSTAX kann Bewerbungen grundlos ablehnen, sollte jedoch stets einen Grund angeben.
\end{enumerate}

\subsection{Werbung}
\begin{enumerate}[(1)]
    \item Die VIRTSTAX verpflichtet sich bei Annahme der Bewerbung, für die angenommene Organisation in dem jeweiligen Kanal zu bewerben und die zukünftige Verfügbarkeit der Aktien dieser Organisation offiziell anzukündigen.
    \item Die Organisation muss dementsprechend auf dem eigenen Server für die VIRTSTAX werben und diese Nachricht in dem Kanal anheften.
\end{enumerate}

\subsection{Teilhaberschaftsbestimmungen}
\begin{enumerate}[(1)]
    \item Geht man die Partnerschaft gemäß \ref{grundb} Abs. 5 ein, so muss man den Eingang dieser Partnerschaft und die damit einhergehenden Änderungen auf dem Discordserver der Organisation ankündigen.
    \item Ebenfalls stimmt man den Teilhaberschaftsbestimmungen zu und der Serverinhaber laut Discord muss diese auf dem Regelkanal hochladen.
    \item Der Partner muss die Organisationsstruktur umgehend so anpassen, dass sie den Teilhaberschaftsbestimmungen entsprechen.
    \item Die Organisation ist verpflichtet, die Teilhaberschaftsbestimmungen stets zu updaten, sobald es Änderungen gibt.
    \item VIRTSTAX ist verpflichtet, Änderungen an den Teilhaberschaftsbestimmungen offiziell anzukündigen.
\end{enumerate}

\subsection{Preisregulation}
\begin{enumerate}[(1)]
    \item Die Festlegung des empfohlenen Preises pro Aktie unterliegt VIRTSTAX und erfordert keine Zustimmung durch die betroffene Organisation.
    \item Der empfohlene Preis muss mindestens dem Aktiengrundwert der Organisation entsprechen.
    \item VIRTSTAX darf die Einhaltung des empfohlenen Preises nicht erzwingen.
    \item Die Organisation muss jederzeit der VIRTSTAX die finanziellen Informationen bei Anfrage zur Verfügung stellen und die VIRTSTAX über Änderungen sofort informieren.
    \item Die finanziellen Informationen, auf die die VIRTSTAX Zugriff haben muss, sind die IBAN, der Name des Empfängers und gegebenenfalls die Email-Adresse des PayPal-Kontos des Zahlungsempfängers, sowie die Kosten, die für die Organisation selbst aufkommen.
    \item Es ist strafbar, die Kosten höher anzugeben, als sie es tatsächlich sind.
    \item Ebenfalls muss man detaillierte Informationen zur Herkunft der Kosten zur Verfügung stellen.
\end{enumerate}

\end{document}